\chapter{Appendix}\label{section-appendix} 
\section{MATLAB Thrust Stabilization-Single tether implementation}
\begin{lstlisting}
% This code is an implementation of Thrust Stabilization
% for single tether
% Hovell, K, and Ulrich, S. (2017)
% "Experimental Validation for Tethered Capture of the Spinning Space Debris"
% AIAA SciTech Forum, AIAA Guidance, Navigation, and Control Conference, Grapevine, Texas. 9 ?13 January
% Paper no AIAA 2017-1049.

% ======
% Nomenclature:
% Jzzt = Moment of inertia, target about z-axis 
% Jzzc = Moment of inertia, chaser about z-axis

% mt = Mass of the target
% mc = Mass of the chaser
% mj = Mass of the junction

% thetac0 = Initial angle of the chaser 
% thetat0 = Initial angle of the target

% TS = Time step

% Lm0= unstretched length of the main tether
% Ls10 = unstretched length of the sub tether 1
% Ls20 = unstretched length of the sub tether 2
% Ls0 = unstretched length of the single tether

% ac = tether attachment point to body chaser with respect to chaser's
% center of mass
% as1 = tether attachment point to body sub tether 1 with respect to
% sub-tether1's center of mass
% as2 = tether attachment point to body sub tether 2 with respect to
% sub-tether2's center of mass
% at = tether attachment point to body target with respect to
% target's center of mass

% csingle = damping coefficient for single tether
% csub = damping coefficient for sub tether

% KPtheta = proportional gain matrix for the theta 
% KDtheta = derivative gain matrix for the theta
% KP = proportional gain matrix for the position
% KD = derivative gain matrix for the postion

% ======
% Define initial conditions and TSS parameters for all simulations and
% experiment:

%in Kg·m^2
Jzzt=0.22; 
Jzzc=0.30; 

%in Kilogram
mt=12.19; 
mc=17.24; 
mj=0.01;

%in Degree
thetac0=180; 
thetat0=0;

% in Second
TS=0.004; 

%in Meter
Lm0=0.28; 
Ls10=0.28; 
Ls20=0.28;  
Ls0=0.54; 
ac=[0.135;-0.009];
as1=[0.110;0.127];
as2=[0.113;-0.094];
at=[0.110;0.016];

% in Newton·second/meter
csingle=3.0; 
csub=0.9;

KPtheta=0.33;
KDtheta=0.56;
KP=[19.1 0;0 19.1];
KD=[32.7 0;0 32.7];

% ======
% Define initial conditions for Thurst Stabilization simulations and experiment
rt0=[0.35;1.19];
rc0=[1.10;1.20];
vt0=[0;0];
vc0=[0;0];
wt0=10; %degree/second
rf=[3.10;1.20];

% ======
% Initialization
% Simulation at time t=2s
% Nothing change in the first two seconds
rt=rt0;
rc=rc0;
vt=vt0;
vc=vc0;
thetat=thetat0;
thetac=thetac0;
Ls=rt+A(thetat)*at-rc-A(thetac)*ac;

% fai is the target tether angle
fai=norm(atand(Ls(2)/Ls(1))-thetat);



wt=wt0;
wc=0;
t=0; % It started to simulate at T=2, which seem it as t=0;
Fs=[0;0];

% AngularMomentum 
% In radius

AngularMomentum = Jzzt * wt/57.3 + Jzzc * wc/57.3 ; 

accelerationc=[0;0];
accelerationt=[0;0];
angularaccelerationt=0;
taot=0;
Fcontrol=[0;0];



for i=1:25000 % i* TS = the time , i is the number of the current time step
%record the data
recordt(i)=t;
recordrc(i,1)=rc(1);
recordrc(i,2)=rc(2);
recordrt(i,1)=rt(1);
recordrt(i,2)=rt(2);
recordthetat(i)=thetat;

recordwt(i)=wt;
recordFs(i,1)=Fs(1);
recordFs(i,2)=Fs(2);

recordLs(i,1)=Ls(1);
recordLs(i,2)=Ls(2);


recordvc(i,1)=vc(1);
recordvc(i,2)=vc(2);

recordvt(i,1)=vt(1);
recordvt(i,2)=vt(2);

recordaccelerationc(i,1)=accelerationc(1);
recordaccelerationc(i,2)=accelerationc(2);
recordaccelerationt(i,1)=accelerationt(1);
recordaccelerationt(i,2)=accelerationt(2);

recordangularaccelerationt(i)=angularaccelerationt;

recordtaot(i)=taot;

recordFcontrol(i,1)=Fcontrol(1);
recordFcontrol(i,2)=Fcontrol(2);
recordfai(i)=fai;

recordAngularMomentum(i) = AngularMomentum ; 

%recordstiffness(i) = stiffness(norm(Ls)-Ls0);

%Ls is the vector which is from chaser to target, pointing at the target
Ls=rt+A(thetat)*at-rc-A(thetac)*ac;

% fai is target tether angle
fai=norm(-atand(Ls(2)/Ls(1))+thetat);


% Calculate the state of the tether 
% Hovell and Ulrich (2017) eqn 19
if (norm(Ls)-Ls0)>0 % tether is tensioned
    Fs=(stiffness(norm(Ls)-Ls0)*(norm(Ls)-Ls0)+...
        dot(csingle*(vt+A(thetat)*[-wt/57.3*at(2);wt/57.3*at(1)]-vc-A(thetac)*[-wc/57.3*ac(2);wc/57.3*ac(1)]),...
        Ls/norm(Ls))).*Ls/norm(Ls);
else % tether is slackness
    Fs=[0;0];
end

% Force on target due to tether tension:
% through center of mass
% referenced to the target's body frame
Fst=(A(thetat))'*Fs;
Fsc=(A(thetac))'*Fs;


% Torque on target due to tether tension:
% about the z-axis of body frame (which is parallel to the z-axis of inertial frame)
% through center of mass target
% referenced to target body frame (the torque referenced to the body is the same as to the inertial frame)
taot=at(2)*Fst(1)-at(1)*Fst(2); % Nm
taoc=ac(1)*Fsc(2)-ac(2)*Fsc(1); % Nm

% The control thrust of chaser vehicle
Fcontrol=KP*(rdes(t)-rc)+KD*(vdes(t)-vc);

% The control torque of chaser vehicle
taocontrol=KPtheta* (180 - thetac) + KDtheta*(0-wc)  ;
accelerationt=(-Fs/mt);
accelerationc=(Fs+Fcontrol)/mc;
angularaccelerationt=taot/Jzzt;
angularaccelerationc=(taoc+taocontrol)/Jzzc;

% position and angle update
rt=rt+vt*TS;
rc=rc+vc*TS;
thetat=thetat+wt*TS;
thetac=thetac+wc*TS;

vt=vt+accelerationt*TS;
wt=wt+angularaccelerationt*TS*(180/pi);%/degree t-1
vc=vc+accelerationc*TS;
wc=wc+angularaccelerationc*TS*(180/pi); %/degree t-1
AngularMomentum = Jzzt * wt/57.3 + Jzzc * wc/57.3;
t=t+TS;
end
recordt=recordt';
recordthetat=recordthetat';
recordwt=recordwt';


recordrc=recordrc';
recordrt=recordrt';
recordvc=recordvc';
recordvt=recordvt';
recordaccelerationc=recordaccelerationc';
recordaccelerationt=recordaccelerationt';
recordangularaccelerationt=recordangularaccelerationt';
recordangularaccelerationt=recordangularaccelerationt';
recordaccelerationc=recordaccelerationc';
recordaccelerationt=recordaccelerationt';
recordtaot=recordtaot';
recordFcontrol=recordFcontrol';

n=1;
figure(n)
plot(recordt,recordwt);
title('Target angular rate')
xlabel('Time,s')
ylabel('Angular rate, deg/s')
n=n+1;

figure(n)
plot(recordt,recordthetat)
title('angle of target')
n=n+1;

figure(n)
plot(recordt,recordrc(1,:))
title(' x position of the chaser')
n=n+1;

figure(n)
plot(recordt,recordrc(2,:))
title('y position of the chaser')
n=n+1;

figure(n)
plot(recordt,recordrt(1,:))
title('x position of the target')
n=n+1;

figure(n)
plot(recordt,recordrt(2,:))
title('y position of the target')
n=n+1;

figure(n)
plot(recordt,mod(recordfai,360))
title('Target tether angle')
xlabel('Time,s')
ylabel('Thether Angle,deg')
n=n+1;

figure(n)
plot(recordt,recordAngularMomentum)
title('Total Angular Momentum')
xlabel('Time,s')
ylabel('Angular Momentum')

n = n + 1;
figure(n)
plot(recordrc(1,:),recordrc(2,:))
title('Track of Chaser')

n = n+1;
figure(n)
plot(recordrt(1,:),recordrt(2,:))
title('Track of Target')

% The function gives the attitude matrix in terms of theta
% cos -sin; sin cos;

function attitude=A(sita)
attitude=[cosd(sita) -sind(sita);sind(sita) cosd(sita)];
end

% calculate the design radius vector of time t
% the goal is to reach the destination in 20s after t=2
function [rdes]=rdes(t)
rc0=[1.10;1.20];
rf=[3.10;1.20];
d=2.*(rf-rc0)./(20*20); %in 20s
[rdes]=(rc0+d.*(t*t/2));
end

% calculate the single tether stiffness(N/m) in terms of
% stretch(m)-independent variable 
% x=(||Ls||-Ls,0)
function stiff=stiffness(x)
originalstiffness=[163.500000000000,114.450000000000,80.2636363636363,51.3857142857143,29.9096000000000,18.8842500000000,17.0724030612245,10.1577253521127,8.06715957446808,7.09840970654628,6.99056826923077,7.22393661971831,8.06497910662824,8.78250400534045,9.58060266159696];
originalstretch=[0.00300000000000000,0.00600000000000001,0.0110000000000000,0.0210000000000000,0.0450000000000000,0.0860000000000001,0.0980000000000001,0.213000000000000,0.329000000000000,0.443000000000000,0.520000000000000,0.639000000000000,0.694000000000000,0.749000000000000,0.789000000000000];
stiff=spline(originalstretch,originalstiffness,x);
end

% calculate the design velocity of time t
% the goal is to reach the destination in 20s after t=2s
function vdes=vdes(t)
rc0=[1.10;1.20];
rf=[3.10;1.20];
d=2.*(rf-rc0)./(20*20); %in 20s
vdes=d*t;
end
\end{lstlisting}

\section{Arduino CSV Streaming Example}\label{appendix-CSV}
\begin{lstlisting}
#include <ArdusatSDK.h>
#include <Wire.h>
#include <Arduino.h>


/*-----------------------------------------------------------------------------
 *  Setup Software Serial to allow for both RF communication and USB communication
 *    RX is digital pin 8 (connect to TX/DOUT of RF Device)
 *    TX is digital pin 9 (connect to RX/DIN of RF Device)
 *-----------------------------------------------------------------------------*/
ArdusatSerial serialConnection(SERIAL_MODE_HARDWARE_AND_SOFTWARE, 8, 9); 

/*-----------------------------------------------------------------------------
 *  Constant Definitions
 *-----------------------------------------------------------------------------*/
 
Acceleration accel;
Gyro gyro;
Magnetic mag;
Orientation orient(accel, mag);

void setup(void) {
  serialConnection.begin(57600);
  
  accel.begin();
  gyro.begin();
  mag.begin();
  orient.begin();
  serialConnection.println("Start of the loop");
  serialConnection.println("The acceleration x y z unit in m/s^2");
  serialConnection.println("The angular rate x y z unit in radian per second");
  serialConnection.println("The orientation roll pitch heading unit in degree");
}

void loop(void) {
  serialConnection.println(accel.readToCSV("Acceleration"));
  serialConnection.println(gyro.readToCSV("Gyro"));
  serialConnection.println(orient.readToCSV("Orientation"));
}

\end{lstlisting}
\section{MATLAB Serial REALTIME PLOTING} 
\begin{lstlisting} 
%% Pre-process
close all;  % close all figures
clear all;  % clear all workspace variables
clc;        % clear the command line
fclose('all'); % close all open files
delete(instrfindall); % reset com port 

%% Constatns
BAUDRATE = 57600; % Need to configure Xbee parts BD i.e. Interface Data Rate
INPUTBUFFER = 51200; %Unit in bytes. Definition:A location that holds all 
% incoming information before it continues to the CPU for processing; 
% Buffers used to store information before it processed


%% Initialize the stripchart
figure_acceleration = figure('Name','Acceleration');
axes_accelerationx = subplot(3,1,1); % Axes Object
xlabel(axes_accelerationx,'Time/ second')
ylabel(axes_accelerationx,'Accelerationx/ m/s^2')

axes_accelerationy = subplot(3,1,2);
xlabel(axes_accelerationy,'Time/ second')
ylabel(axes_accelerationy,'Accelerationy/ m/s^2')

axes_accelerationz = subplot(3,1,3);
xlabel(axes_accelerationz,'Time/ second')
ylabel(axes_accelerationz,'Accelerationz/ m/s^2')

h_ax = animatedline(axes_accelerationx,'MaximumNumPoints',Inf,'Marker','o','Color','red'); % animated line object
h_ay = animatedline(axes_accelerationy,'MaximumNumPoints',Inf,'Marker','+','Color','blue');
h_az = animatedline(axes_accelerationz,'MaximumNumPoints',Inf,'Marker','*','Color','green');

figure_gyro = figure('Name','Gyro');
axes_gyrox = subplot(3,1,1);
xlabel(axes_gyrox,'Time/ second')
ylabel(axes_gyrox,'Gyrox/ rad/s')

axes_gyroy = subplot(3,1,2);
xlabel(axes_gyroy,'Time/ second')
ylabel(axes_gyroy,'Gyroy/ rad/s')

axes_gyroz = subplot(3,1,3);
xlabel(axes_gyroz,'Time/ second')
ylabel(axes_gyroz,'Gyroz/ rad/s')

h_gx = animatedline(axes_gyrox,'MaximumNumPoints',Inf,'Marker','o','Color','red');
h_gy = animatedline(axes_gyroy,'MaximumNumPoints',Inf,'Marker','+','Color','blue');
h_gz = animatedline(axes_gyroz,'MaximumNumPoints',Inf,'Marker','*','Color','green');

figure_orientation = figure('Name','Orientation');
axes_roll = subplot(3,1,1);
xlabel(axes_roll,'Time/ second')
ylabel(axes_roll,'Roll/ deg')

axes_pitch = subplot(3,1,2);
xlabel(axes_pitch,'Time/ second')
ylabel(axes_pitch,'Pitch/ deg')

axes_heading = subplot(3,1,3);
xlabel(axes_heading,'Time/ second')
ylabel(axes_heading,'Heading/ deg')

h_roll = animatedline(axes_roll,'MaximumNumPoints',Inf,'Marker','o','Color','red');
h_pitch = animatedline(axes_pitch,'MaximumNumPoints',Inf,'Marker','+','Color','blue');
h_heading =animatedline(axes_heading,'MaximumNumPoints',Inf,'Marker','*','Color','green');

% Index 
i = 1;
j = 1;
k = 1;

% Record the data
% 10000 Row data
record_acceleration = zeros(10000,4);
record_gyro = zeros(10000,4);
record_orientation = zeros(10000,4);

%% Create a serial object
board = serial('COM4','BaudRate',BAUDRATE,'DataBits',8);
% Related with the used COM, can be different 
% COM4 for wireless, COM3 for wire
% BAUDRATE unit: bits per second, e.g. 9600 means 9600 bits per second,
% i.e. 1200 bytes per second i.e. 1.2 KB/s or 1.17KB/s


% Set serial port buffer
set(board,'InputBufferSize',INPUTBUFFER);
% InputBufferSize: total number of bytes that can be stored in the input
% buffer during a read operation. The read operation is terminated if the
% amount of data stored in the input buffer equals the InputBufferSize

fopen(board);
while(1)
    a = fgetl(board);% a:character vector
    b = textscan(a,'%u32 %s %f %f %f %u32','Delimiter',',');
    
    if strcmp(char(b{2}),'Acceleration')
    
        time = double(b{1})/1000; % in second
        ax = b{3}; 
            addpoints(h_ax,time,ax);
            drawnow limitrate
             
        ay = b{4};
            addpoints(h_ay,time,ay);
            drawnow limitrate
            
        az = b{5};
            addpoints(h_az,time,az);
            drawnow limitrate
            
        record_acceleration(i,:)=[time ax ay az];
        i=i+1;
        
    elseif strcmp(char(b{2}),'Gyro')
        time = double(b{1})/1000; % in second
        gx = b{3}; 
            addpoints(h_gx,time,gx);
            drawnow limitrate
            
        gy = b{4};
            addpoints(h_gy,time,gy);
            drawnow limitrate
        gz = b{5};
            addpoints(h_gz,time,gz);
            drawnow limitrate
            
        record_gyro(j,:)=[time gx gy gz];
        j=j+1;
        
    elseif strcmp(char(b{2}),'Orientation')
        time = double(b{1})/1000; % in second
        roll = b{3};
            addpoints(h_roll,time,roll);
            drawnow limitrate
            
        pitch = b{4};
            addpoints(h_pitch,time,pitch);
            drawnow limitrate
            
        heading = b{5};
            addpoints(h_heading,time,heading);
            drawnow limitrate
            
        record_orientation(k,:) = [time roll pitch heading];
        k = k+1;
    end 

end 
\end{lstlisting} 
\newpage    
