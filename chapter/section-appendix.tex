\chapter{Appendix}\label{section-appendix}
\section{MATLAB}



\section{Arduino CSV Streaming Example}
\verb{
#include <ArdusatSDK.h>
#include <Wire.h>
#include <Arduino.h>


/*-----------------------------------------------------------------------------
 *  Setup Software Serial to allow for both RF communication and USB communication
 *    RX is digital pin 8 (connect to TX/DOUT of RF Device)
 *    TX is digital pin 9 (connect to RX/DIN of RF Device)
 *-----------------------------------------------------------------------------*/
ArdusatSerial serialConnection(SERIAL_MODE_HARDWARE_AND_SOFTWARE, 8, 9); 

/*-----------------------------------------------------------------------------
 *  Constant Definitions
 *-----------------------------------------------------------------------------*/
 
Acceleration accel;
Gyro gyro;
Magnetic mag;
Orientation orient(accel, mag);


void setup(void) {
  serialConnection.begin(57600);
  
  accel.begin();
  gyro.begin();
  mag.begin();
  orient.begin();
  serialConnection.println("Start of the loop");
  serialConnection.println("The acceleration x y z unit in m/s^2");
  serialConnection.println("The angular rate x y z unit in radian per second");
  serialConnection.println("The orientation roll pitch heading unit in degree");
}

void loop(void) {
  serialConnection.println(accel.readToCSV("Acceleration"));
  serialConnection.println(gyro.readToCSV("Gyro"));
  serialConnection.println(orient.readToCSV("Orientation"));
}
}

\section{MATLAB Serial REALTIME PLOTING}
%
%%% Pre-process
%close all;  % close all figures
%clear all;  % clear all workspace variables
%clc;        % clear the command line
%fclose('all'); % close all open files
%delete(instrfindall); % reset com port 
%
%%% Constatns
%BAUDRATE = 57600; % Need to configure Xbee parts BD i.e. Interface Data Rate
%INPUTBUFFER = 51200; %Unit in bytes. Definition:A location that holds all 
%% incoming information before it continues to the CPU for processing; 
%% Buffers used to store information before it processed
%
%
%%% Initialize the stripchart
%figure_acceleration = figure('Name','Acceleration');
%axes_accelerationx = subplot(3,1,1); % Axes Object
%xlabel(axes_accelerationx,'Time/ second')
%ylabel(axes_accelerationx,'Accelerationx/ m/s^2')
%
%axes_accelerationy = subplot(3,1,2);
%xlabel(axes_accelerationy,'Time/ second')
%ylabel(axes_accelerationy,'Accelerationy/ m/s^2')
%
%axes_accelerationz = subplot(3,1,3);
%xlabel(axes_accelerationz,'Time/ second')
%ylabel(axes_accelerationz,'Accelerationz/ m/s^2')
%
%h_ax = animatedline(axes_accelerationx,'MaximumNumPoints',Inf,'Marker','o','Color','red'); % animated line object
%h_ay = animatedline(axes_accelerationy,'MaximumNumPoints',Inf,'Marker','+','Color','blue');
%h_az = animatedline(axes_accelerationz,'MaximumNumPoints',Inf,'Marker','*','Color','green');
%
%figure_gyro = figure('Name','Gyro');
%axes_gyrox = subplot(3,1,1);
%xlabel(axes_gyrox,'Time/ second')
%ylabel(axes_gyrox,'Gyrox/ rad/s')
%
%axes_gyroy = subplot(3,1,2);
%xlabel(axes_gyroy,'Time/ second')
%ylabel(axes_gyroy,'Gyroy/ rad/s')
%
%axes_gyroz = subplot(3,1,3);
%xlabel(axes_gyroz,'Time/ second')
%ylabel(axes_gyroz,'Gyroz/ rad/s')
%
%h_gx = animatedline(axes_gyrox,'MaximumNumPoints',Inf,'Marker','o','Color','red');
%h_gy = animatedline(axes_gyroy,'MaximumNumPoints',Inf,'Marker','+','Color','blue');
%h_gz = animatedline(axes_gyroz,'MaximumNumPoints',Inf,'Marker','*','Color','green');
%
%figure_orientation = figure('Name','Orientation');
%axes_roll = subplot(3,1,1);
%xlabel(axes_roll,'Time/ second')
%ylabel(axes_roll,'Roll/ deg')
%
%axes_pitch = subplot(3,1,2);
%xlabel(axes_pitch,'Time/ second')
%ylabel(axes_pitch,'Pitch/ deg')
%
%axes_heading = subplot(3,1,3);
%xlabel(axes_heading,'Time/ second')
%ylabel(axes_heading,'Heading/ deg')
%
%h_roll = animatedline(axes_roll,'MaximumNumPoints',Inf,'Marker','o','Color','red');
%h_pitch = animatedline(axes_pitch,'MaximumNumPoints',Inf,'Marker','+','Color','blue');
%h_heading =animatedline(axes_heading,'MaximumNumPoints',Inf,'Marker','*','Color','green');
%
%% Index 
%i = 1;
%j = 1;
%k = 1;
%
%% Record the data
%% 10000 Row data
%record_acceleration = zeros(10000,4);
%record_gyro = zeros(10000,4);
%record_orientation = zeros(10000,4);
%
%%% Create a serial object
%board = serial('COM4','BaudRate',BAUDRATE,'DataBits',8);
%% Related with the used COM, can be different 
%% COM4 for wireless, COM3 for wire
%% BAUDRATE unit: bits per second, e.g. 9600 means 9600 bits per second,
%% i.e. 1200 bytes per second i.e. 1.2 KB/s or 1.17KB/s
%
%
%% Set serial port buffer
%set(board,'InputBufferSize',INPUTBUFFER);
%% InputBufferSize: total number of bytes that can be stored in the input
%% buffer during a read operation. The read operation is terminated if the
%% amount of data stored in the input buffer equals the InputBufferSize
%
%fopen(board);
%while(1)
%    a = fgetl(board);% a:character vector
%    b = textscan(a,'%u32 %s %f %f %f %u32','Delimiter',',');
%    
%    if strcmp(char(b{2}),'Acceleration')
%    
%        time = double(b{1})/1000; % in second
%        ax = b{3}; 
%            addpoints(h_ax,time,ax);
%            drawnow limitrate
%             
%        ay = b{4};
%            addpoints(h_ay,time,ay);
%            drawnow limitrate
%            
%        az = b{5};
%            addpoints(h_az,time,az);
%            drawnow limitrate
%            
%        record_acceleration(i,:)=[time ax ay az];
%        i=i+1;
%        
%    elseif strcmp(char(b{2}),'Gyro')
%        time = double(b{1})/1000; % in second
%        gx = b{3}; 
%            addpoints(h_gx,time,gx);
%            drawnow limitrate
%            
%        gy = b{4};
%            addpoints(h_gy,time,gy);
%            drawnow limitrate
%        gz = b{5};
%            addpoints(h_gz,time,gz);
%            drawnow limitrate
%            
%        record_gyro(j,:)=[time gx gy gz];
%        j=j+1;
%        
%    elseif strcmp(char(b{2}),'Orientation')
%        time = double(b{1})/1000; % in second
%        roll = b{3};
%            addpoints(h_roll,time,roll);
%            drawnow limitrate
%            
%        pitch = b{4};
%            addpoints(h_pitch,time,pitch);
%            drawnow limitrate
%            
%        heading = b{5};
%            addpoints(h_heading,time,heading);
%            drawnow limitrate
%            
%        record_orientation(k,:) = [time roll pitch heading];
%        k = k+1;
%    end 
%
%end
\newpage
    
