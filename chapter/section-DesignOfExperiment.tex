\chapter{Experiment Design}\label{sec-DOE}
This chapter talks about the Experiment set up.
\section{ArduSat}
The HKUST CYT Lab has a lot of ArduSat DemosSats\cite{geeroms2015ardusat,peters2016ardusat}. ArduSat is an Arduino based Nanosatellite, based on the CubeSat standard. It contains a set of Arduino boards and sensors. The general public will be allowed to use these Arduinos and sensors for their own creative purposes while they are in space. 
\begin{figure}[ht]
\centering
\includegraphics[width=6cm]{fig/DOE/Demosat}
\caption{ArduSat DemoSat}
\end{figure}
The DemoSat is designed to the CubeSat One Unit (1U) standard. The 3D printed frame comes fully assembled with standoffs and see-through acrylic platforms. Each DemoSat also includes all of the components included in the Because Learning Sensor Kit. The Because Learning Sensor Kit includes:
\begin{enumerate}
\item Micro controller programmed w/ Arduino 
\item Because Learning 'Sensor board' sensors 
\begin{itemize}
\item Accelerometer
\item Gyroscope
\item Magnetometer
\end{itemize}
\end{enumerate} 
Also the DemoSat includes two wireless radio frequency (RF) radios. They are Digi XBee 2.4Ghz modules. One is connected inside the DemoSat and the second is connect via USB to laptop. By doing this it enables the DemoSatellite to communicate up to 1500 meter from the computer.
It can get our interested acceleration, angular rate and orientation of the Satellites at real time wirelessly.
\subsection{Arduino Data Acquisition}
Our interested physical quantities are acceleration, angular rate and Orientation. By using the ArdusatSDK\cite{ardusatsdk}, acceleration can be read from Sensor LSM303 triple-axis accelerometer and Angular rates can be read from Sensor L3GD20 three-axis gyroscope. The orientation i.e. roll, pitch and heading, can be derived from readings from the Acceleration and Magnetic Sensors LSM303-Triple-axis Accelerometer plus Magnetometer (Compass) Board (Figure \ref{compass board}).

\begin{figure}[!b]
\centering
\includegraphics{fig/DOE/LSM}
\caption{LSM303 Triple-axis Accelerometer plus Magnetometer (Compass) Board}\label{compass board}
\end{figure}

For the reason of real-time plotting simplicity, the Comma-separated values(CSV) format which provides the timestamp of action, is adopted. The CSV streaming Arduino code is written at Appendix \ref{appendix-CSV}. The CSV format is: \textbf{timestamp,sensorName,values,...,checksum}.

For example, one of the result is like this(during around 100ms):

\begin{quote}
\centering
...

33,Acceleration,0.122,-0.022,-8.894,1217

44,Gyro,-0.177,-0.999,-0.142,416

53,Orientation,-179.737,-0.759,7.834,991

64,Acceleration,0.074,-0.062,-8.863,1217

74,Gyro,-0.181,-1.074,-0.112,416

83,Orientation,-179.598,-0.480,7.808,992

95,Acceleration,0.033,-0.067,-8.944,1217

105,Gyro,-0.216,-0.988,-0.177,416

115,Orientation,-179.739,-0.384,7.846,992

126,Acceleration,0.045,-0.010,-8.920,1217

137,Gyro,-0.110,-0.318,-0.061,417

...
\end{quote}

The time-stamp is the time value from which the program was operated, in millisecond. The last number is checksum which used for check the communication.
The baud rate is set as 5700 as softwareSerial does not appear to work reliably above 57600 baud.\cite{ardusatsdk}
When using baud rate as 57600, the update frequency of physical quantities are 32Hz, which means every second we can get 32 groups of acceleration, angular rate and orientation. The unit of baud rate is bits per second.i.e. 57600 bits per second equals 7.2KB/s\cite{dong1994device}.

\subsection{Xbee modules configuration}
\begin{figure}[!h]
\centering
\includegraphics{fig/DOE/XbeeShield}
\caption{Xbee part with Shield}
\end{figure}
Digi XCTU\cite{DigiCite} is the software used to configure the Xbee modules. In our case, two Ardusat Demosat should send the real time data to the computers separately.The two XBee parts are on Ardusat A and Ardusat B The other two Xbee parts are plugged in shields and the shields are connected to Computer A and Computer B respectively. All the Xbee parts should be set carefully\cite{XbeeSetting}. 
\begin{table}[!h]
\renewcommand\arraystretch{2}
	\begin{center}
	\caption{Xbee parts setting}\label{Xbee-configuration}
	\begin{tabular}{|c|c|c|c|c|}
	\hline
	\backslashbox{\textbf{Setting}}{\textbf{Xbee part}} & \textbf{A shield} & \textbf{On Ardusat A} & \textbf{B shield} & \textbf{On ArduSat B}\\ \hline
	Channel & \multicolumn{2}{c|}{C} &\multicolumn{2}{c|}{17}  \\ \hline
	Personal area network ID & \multicolumn{2}{c|}{3332}& \multicolumn{2}{c|}{83D5}\\ \hline
	Destination Address High & 0 & 0 & 0 & 13A200\\ \hline
	Destination Address Low & 35 & 0 & 1234&4163E25D\\ \hline
	16-bit Source Address & 0 & 35 & 4321 & 1532\\ \hline
	Serial Number High & \multicolumn{4}{c|}{13A200}\\ \hline
	Serial Number Low & 4167BD1E & 4163E2BA & 4163E25D & 4167BD2E\\ \hline
	Interface Data Rate	 & \multicolumn{4}{c|}{57600}\\ \hline
	\end{tabular}	
	\end{center}
\end{table}
The channel controls the frequency band that Xbee communicates over. XBee's operate on the 2.4GHZ 802.15.4 band, and the channel further calibrates the operating frequency within that band. The two XBee modules having on the same network operates on the same channel, like Figure \ref{xbeenetworkexample}.  The two networks are using different channels in case they interact with each other.

Personal area network ID (PAN ID) is some hexadecimal value between 0 and 0xFFFF. Similar to the channel, the two Xbee modules in the same network must have the same network ID and the different network should have different PAN ID.

 Each XBee in a network should be assigned a 16-bit address (again between 0 and 0xFFFF), which is referred to as MY address, or the "source" address. Another setting, the destination address, determines which source address an XBee can send data to. For one XBee to be able to send data to another, it must have the same destination address as the other XBee's source.

Each XBee has a unique 16-bit Source Address i.e. MY address.

In our case, the task is that On Ardusat A XBee should only send data to to A shield Xbee and On Ardusat B Xbee should only send data to B shield Xbee.

There are two way to configure destination address
\begin{enumerate}
\item Leave Destination Address High set to 0, and set Destination Address Low to the MY address of the receiving XBee.\label{method1}

\item Set Destination Address High to the Serial Number High (SH) and Destination Address Low to the Serial Number Low (SL) of your destination XBee.\label{method2}
\end{enumerate}
To assure that each network has high fidelity and the communication interference between A network and B network is as low as possible, the A network is only using method \ref{method1} and the B network is using method \ref{method2}. A network's configuration obeys method \ref{method1} and disobeys method \ref{method2}. B network's configuration obeys method \ref{method2} and goes against method \ref{method1}.

The key is to reduce communication interference when two satellites are moving especially when they are quite close.The final configuration is showed at Table \ref{Xbee-configuration}.

\begin{figure}[ht]
\centering
\includegraphics[width = 0.5\textwidth]{fig/DOE/Network}
\caption{Xbee-networks example}\label{xbeenetworkexample}
\end{figure}

\subsection{Real-time data streaming to MATLAB via USB}
Since the Xbee modules can send data wirelessly to the shield module, the Arduino Serial Monitor is able to receive and pop out the streaming txt data. However, Arduino cannot plot the received data at real time and save the data synchronously.

To plot the real-time streaming data and record them we need to use MATLAB\cite{manalo2012usb}. MATLAB serial can create the serial port object. The Xbee shield is connected to computer via USB and using serial port to do data transmission.

The MATLAB code in Appendix \ref{appendix-realtime} gives the functionality. This MATLAB file reads the serial post, plots the three physical quantities and saves the data into record matrix.       

\begin{figure}[H]
\centering
\includegraphics[width=\textwidth]{fig/DOE/acceleration}
\includegraphics[width=\textwidth]{fig/DOE/gyro}
\includegraphics[width=\textwidth]{fig/DOE/orientation}
\caption{MATLAB plot real-time streaming figure}
\end{figure}

\section{Testbed fabrication}
Utilizing Porous Media Technology Air Bushings deliver low coefficients of friction and high motion-control resolution for less-expensive road. The ideal components for building frictionless linear motion, Air Bearings were the original standard, off-the-shelf, porous media air bearing product line. 

The first version of Air Bushing Planar Torque Free Motion simulator(Hong Kong indigenous nanosatellite engineering test facilities), designed by Dr. Pok Wang KWAN is showing at Figure \ref{MotionSimulator} The experimental has been carefully engineered to reduce the cost while it intended to provide good performance. The air bushing floating platform shall provide low friction translational motion on horizontal plane(two degrees of freedom).

\begin{figure}[H]
\centering
\includegraphics[width=\textwidth]{fig/DOE/Original3DView}
\caption{Air Bushing Planar Torque Free Motion simulator}\label{MotionSimulator}
\end{figure}

\begin{figure}[H]
\centering
\includegraphics[width=\textwidth]{fig/DOE/3viewdrawing}
\caption{Air Bushing Planar Torque Free Motion simulator-Three View Drawing}
\end{figure}

However, the Air Bushing Torque Free Motion simulator has a fatal problem. As the weight of top floor air bushings and top floor rods are bear by the bottom floor air bushings and rods, the mass of inertia along two axes are different. If two forces with same magnitude exerted on the same point of action along two vertical axes, the top floor air bushings will generate much bigger acceleration than the bottom floor air bushings since the motion along that way is much easier.


\begin{figure}[H]
\centering
\begin{minipage}[t]{0.48\textwidth}
\centering
\includegraphics[width = 6cm]{fig/DOE/Rightforce}
\caption{Exerted force Right View}
\end{minipage}
\begin{minipage}[t]{0.48\textwidth}
\centering
\includegraphics[width = 6cm]{fig/DOE/ForwardForce}
\caption{Exerted force Front View}\label{DOE-force}
\end{minipage}
\end{figure}


In Solidworks, using Motion study modules, the simulation result(Figure \ref{Resulte of analysis}) of Motion Analysis also support this argument. As we can see the blue line represent the acceleration(2) of bottom floor, around 81 $mm/sec^2$ which is much much smaller than the acceleration of top floor. The top floor air bushing has the acceleration more than 1300 $mm/sec^2$. The oscillation of acceleration means the air bushing of top floor hits the border and got reversal small acceleration. The exerted force have the same magnitude and vertical direction. One is along the long thick rod and the other is along the short thin rod.


\begin{figure}[H]
\centering
\includegraphics[width=\linewidth]{fig/DOE/MotionAnalysis}
\caption{Solidworks Motion Analysis}\label{Resulte of analysis} 
\end{figure} 

In theory the problem can be solved by using compensation mechanism. If we add another pair of Air Bushing Torque Free Motion simulator at the top of the original one and let them be vertical, then we use some physical connection such as rod or bar to connect these two pairs of simulator. The top plane and the bottom plane are fixed. The mass of Inertia along two axes now become the same and the problem can be solved. 

\begin{figure}[H]
\centering
\begin{minipage}[t]{0.48\textwidth}
\centering
\includegraphics[width=6cm]{fig/DOE/FrontView}
\end{minipage}
\begin{minipage}[t]{0.48\textwidth}
\centering
\includegraphics[width=6cm]{fig/DOE/RightView}
\end{minipage}
\caption{Compensation configuration conception}
\end{figure}

Due to the weight of the fixing platform is too heavy and the additional need of rotation degree of freedom,the Suspension/Hanging configuration is used. A shroud bracket is connected with the moving air bushing mounting blocks and moves together. The Cubusat is connected to shroud bracket using a string, which provide the rotational degree of freedom. 

\begin{figure}[H]
\centering
\includegraphics[width=0.3\linewidth]{fig/DOE/Hang}
\caption{Hanging configuration} 
\end{figure} 

\newpage
