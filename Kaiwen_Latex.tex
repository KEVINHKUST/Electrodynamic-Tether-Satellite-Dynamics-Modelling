%%%%%%%%%%%%%%%%%%%%%%%%%%%%%%%%%%%%%%%%%%%%%%%%%%%%%%%%%%%%%%%%%%%%%%%%%
%                                                                       %
% ustthesis_test.tex: A template file for usage with ustthesis.cls      %
%                                                                       %
%%%%%%%%%%%%%%%%%%%%%%%%%%%%%%%%%%%%%%%%%%%%%%%%%%%%%%%%%%%%%%%%%%%%%%%%%

\documentclass{ustthesis}

\usepackage{mathpazo,amsmath,amssymb,epsfig,enumerate,bbm,calc,color,ifthen,capt-of} % original was times, but I think it's ugly; we use the same as IEEE CompSoc
\usepackage{algorithm}
\usepackage{booktabs}
\usepackage{tabularx}
\usepackage[noend]{algorithmic}
\usepackage[center]{subfigure}
\usepackage{color,graphicx}
\newtheorem{proof}{Proof}
\usepackage{hyperref} % for better viewing experience  -- added by alan
\usepackage[margin=25mm,textheight=247mm,textwidth=145mm]{geometry}

% Alan: begin the font trial
% Euler for math | Palatino for rm | Helvetica for ss | Courier for tt
\renewcommand{\rmdefault}{ppl} % rm
%\linespread{1.05}        % Palatino needs more leading
\usepackage[scaled]{helvet} % ss
\usepackage{courier} % tt 
\usepackage{eulervm} % a better implementation of the euler package (not in gwTeX)
\normalfont
\usepackage[T1]{fontenc}
% Alan: end the font trial

\newcommand{\red}[1]{#1}
\newcommand{\tab}[1]{\hspace{3mm}}

% \usepackage{latexsym}
    % Use the "latexsym" package when encountering the following error:
    %   ! LaTeX Error: Command \??? not provided in base LaTeX2e.
% \usepackage{epsf}
    % Use the "epsf" package for including EPS files.

%%%%%%%%%%%%%%%%%%%%%%%%%%%%%%%%%%%%%%%%%%%%%%%%%%%%%%%%%%%%%%%%%%%%%%%%%
%                                                                       %
% Preambles. DO NOT ERASE THEM. Change to suite your particular purpose.%
%                                                                       %
%%%%%%%%%%%%%%%%%%%%%%%%%%%%%%%%%%%%%%%%%%%%%%%%%%%%%%%%%%%%%%%%%%%%%%%%%

\title{Electrodynamic Tether Satellite Dynamics Modelling}  % Title of the thesis.
\author{Kaiwen~Chen}     % Author of the thesis.
\degree{\MSc}             % Degree for which the thesis is.
\subject{Aeronautical Engineering}      % Subject of the Degree.
\department{Mechanical and Aerospace Engineering}       % Department to which the thesis
                    % is submitted.
\advisor{Prof.~Xun~HUANG}     % Supervisor.
\depthead{Prof.~Christopher~Yu-Hang~CHAO}    % department head.
\defencedate{2018}{05}{29}      % \defencedate{year}{month}{day}.

% NOTE:
%   According to the sample shown in the guidelines, page number is
%   placed below the bottom margin.  However, if the author prefers
%   the page number to be printed above the bottom margin, please
%   activate the following command.

% \PNumberAboveBottomMargin

\begin{document}

%%%%%%%%%%%%%%%%%%%%%%%%%%%%%%%%%%%%%%%%%%%%%%%%%%%%%%%%%%%%%%%%%%%%%%%%%
%                                                                       %
% Now the actual Thesis. The order of output MUST be followed:          %
%                                                                       %
%    1) TITLEPAGE                                                       %
%                                                                       %
% The \maketitle command generates the Title page as well as the        %
% Signature page.                                                       %
%                                                                       %
%%%%%%%%%%%%%%%%%%%%%%%%%%%%%%%%%%%%%%%%%%%%%%%%%%%%%%%%%%%%%%%%%%%%%%%%%

\maketitle

%%%%%%%%%%%%%%%%%%%%%%%%%%%%%%%%%%%%%%%%%%%%%%%%%%%%%%%%%%%%%%%%%%%%%%%%%
%                                                                       %
%     2) DEDICATION (Optional)                                          %
%                                                                       %
% The \dedication and \enddedication commands are optional. If          %
% specified it generates a page for dedication.                         %
%
%%%%%%%%%%%%%%%%%%%%%%%%%%%%%%%%%%%%%%%%%%%%%%%%%%%%%%%%%%%%%%%%%%%%%%%%%

% \dedication
% This is an optional section.
% \enddedication

%%%%%%%%%%%%%%%%%%%%%%%%%%%%%%%%%%%%%%%%%%%%%%%%%%%%%%%%%%%%%%%%%%%%%%%%%
%                                                                       %
%     3) ACKNOWLEDGMENTS                                                %
%                                                                       %
% \acknowledgments and \endacknowledgments defines the                  %
% Acknowledgments of the author of the Thesis.                          %
%                                                                       %
%%%%%%%%%%%%%%%%%%%%%%%%%%%%%%%%%%%%%%%%%%%%%%%%%%%%%%%%%%%%%%%%%%%%%%%%%

\acknowledgments
I would never have completed this work without the help from many people. First of all, I thank my advisor, Professor Xun Huang, for his kind mentoring, helpful advices, and encouragement. I have got generous material support, well-condition laboratory experiences and engineering-oriented practice.  

I thank the members of my thesis committee, Professor Xin Zhang for his insightful comments on improving this work. 

I thank my colleagues in HKUST -- Huanxian Bu, Zheng Liu, Wei Feng, Jiafeng Wu, Jingwen Guo, Jiaqi Song and many others. We have finished several deadlines and projects as a team. In daily life, we have been good friends and enjoy many activities in school campus. Without them, my graduate study in HKUST would not be so colorful. 

I give my special thanks to Dr.Pok Wang KWAN who spent nearly a whole year instructing me on how to do research. He gave the project a lot of remarkable ideas using his sufficient engineering background and rich industry experience. When everything goes more and more complicated, his clear mind and intuition always work quite well. I learned how to proceed an engineering research project and how to analyse problems logically and incrementally from his face to face teaching.

I thank Github user Cheedoong and wenbinf for their contribution and sharing of the HKUST Latex thesis Template. This thesis is writing in Latex based on that template-ustthesis Latex class. It saves me a lot of time to do typesetting and let me focus on the contents.   

I thank my parents for their support and encouragement. 

Hope this thesis will not be the end of my academia, I hope the first half sentence is not just a hope.

\endacknowledgments

%%%%%%%%%%%%%%%%%%%%%%%%%%%%%%%%%%%%%%%%%%%%%%%%%%%%%%%%%%%%%%%%%%%%%%%%%
%                                                                       %
%     4) TABLE OF CONTENTS                                              %
%                                                                       %
%%%%%%%%%%%%%%%%%%%%%%%%%%%%%%%%%%%%%%%%%%%%%%%%%%%%%%%%%%%%%%%%%%%%%%%%%

\tableofcontents

%%%%%%%%%%%%%%%%%%%%%%%%%%%%%%%%%%%%%%%%%%%%%%%%%%%%%%%%%%%%%%%%%%%%%%%%%
%                                                                       %
%     5) LIST OF FIGURES (If Any)                                       %
%                                                                       %
%%%%%%%%%%%%%%%%%%%%%%%%%%%%%%%%%%%%%%%%%%%%%%%%%%%%%%%%%%%%%%%%%%%%%%%%%

\listoffigures

%%%%%%%%%%%%%%%%%%%%%%%%%%%%%%%%%%%%%%%%%%%%%%%%%%%%%%%%%%%%%%%%%%%%%%%%%
%                                                                       %
%     6) LIST OF TABLES (If Any)
%                                                                       %
%%%%%%%%%%%%%%%%%%%%%%%%%%%%%%%%%%%%%%%%%%%%%%%%%%%%%%%%%%%%%%%%%%%%%%%%%

\listoftables

%%%%%%%%%%%%%%%%%%%%%%%%%%%%%%%%%%%%%%%%%%%%%%%%%%%%%%%%%%%%%%%%%%%%%%%%%
%                                                                       %
%     7) ABSTRACT                                                       %
%                                                                       %
% \abstract and \endabstract are used to define a short Abstract for    %
% the Thesis.                                                           %
%                                                                       %
%%%%%%%%%%%%%%%%%%%%%%%%%%%%%%%%%%%%%%%%%%%%%%%%%%%%%%%%%%%%%%%%%%%%%%%%%

\begin{abstract}
Abstract
\end{abstract}


%%%%%%%%%%%%%%%%%%%%%%%%%%%%%%%%%%%%%%%%%%%%%%%%%%%%%%%%%%%%%%%%%%%%%%%%%
%                                                                       %
%     8) The Actual Contents                                            %
%                                                                       %
% The command \chapters MUST BE USED to ensure that the entire content  %
% of the Thesis is double-spaced (in version 1.0).                      %
%                                                                       %
% However, in version 2.0, \chapters will be automatically added in     %
% the beginning of the first chapter.                                   %
%                                                                       %
%%%%%%%%%%%%%%%%%%%%%%%%%%%%%%%%%%%%%%%%%%%%%%%%%%%%%%%%%%%%%%%%%%%%%%%%%

%%\chapters         % Not necessary with ustthesis.cls (v2.0).

%%%%%%%%%%%%%%%%%%%%%%%%%%%%%%%%%%%%%%%%%%%%%%%%%%%%%%%%%%%%%%%%%%%%%%%%%
%                                                                       %
% Each chapter is defined via the \chapter command. The usual sectional %
% commands of LaTeX are also available.                                 %
%                                                                       %
%%%%%%%%%%%%%%%%%%%%%%%%%%%%%%%%%%%%%%%%%%%%%%%%%%%%%%%%%%%%%%%%%%%%%%%%%

\input{chapter/section-Motivation}
\chapter{Literature Review}\label{sec-literature}
\section{Space tether past missions}
\begin{enumerate} 


\item{NASA}
 
NASA has been developing tether technology for space applications since the 1960s, including electrodynamic tether propulsion, the Propulsive Small Expendable Deployer System (ProSEDS) flight experiment, ” Hanging” momentum exchange tethers, rotating momentum exchange tethers and tethers supporting scientific space research. 

\item{NASA-Gemini XI} 

The Gemini XI was a manned spaceflight in NASA’s Gemini program, launched on September 12,1966. One of the main objectives was related to tethers. It was synoptic terrain photography and a tethered vehicle test. The objectives were all achieved. 

\item{NASA-Gemini XII} 
   
The Gemini XII was a manned spaceflight in NASA’s Gemini program launched on November 11, 1966. One of its major objectives was tethered vehicle evaluation. The objective was achieved.
\item{Tethered Payload Experiment (TPE)}

TPE-1mission was launched on January 16, 1980. Its plan was to deploy 400 metres of cable, but its deployed cable was about 38 metres. The TPE-2 mission was launched on 29 January, 1981, and its tether was deployed to a distance about 65 metres. In 1983, the TPE-3, which was also called CHARGE-1, had a length of about 500m. As the deployment system was improved, the tether deployed to its full length of 418 meters, and the tether was also found to act as a radio antenna for the electrical current through the cable. CHARGE-2 was carried out as an international program between Japan and the USA using a NASA sounding rocket at White Sands Missile Range, in December 1985, its tether deployed to a length of 426 metres.
CHARGE-2B tethered rocket mission was launched in 1992 by NASA with a Black Brant V rocket. The mission was to generate electromagnetic waves by modulating the electron beam. The tether was fully deployed over 400 meters and the experiments all worked as planned.
\item{NASA and NDRE-MAIMIK}

	The tether length of MAIMIK was about 400m. The mission was designed to study the charging of an electron-beam emitting payload using a tethered mother-daughter payload configuration. 

\item{US Air Force Geophysics Lab- Echo-7}

Designed to study the artificial electron beam propagation along magnetic field lines in space, the mission planned to study how the artificial electron beam propagates along magnetic field lines in space.
\item{OEDIPUS}

OEDIPUS-A. The conducting tether was deployed over 985 metres during the flight of a Black Brant sounding rocket in to the auroral ionosphere.
OEDIPUS-C. The following OEDIPUS mission was the OEDIPUS-C tethered payload mission, which was launched in 1995 with an 1174-metre deployed tether, and a Tether Dynamics Experiment (TDE) was also included as a part of the OEDIPUS-C.
\item{Tethered Satellite System (TSS)}

The first orbital flight experiment with a long tether was the Tethered Satellite System (TSS) mission, launched on the Space Shuttle in July 1992. The Tethered Satellite System-1 (TSS-1) was flown during STS-46, aboard the Space Shuttle Atlantis, from July 31 to August 8, 1992. The TSS-1 mission discovered a lot about the dynamics of the tethered system. Although the satellite was deployed only 260 metres, it was able to show that the tether could be deployed, controlled, and retrieved, and that the TSS was easy to control, and even more stable than predicted. The TSS was an electrodynamic tether, its deployment mechanism jammed resulting in tether sever and less than 1000 metres of deployment. The objectives of TSS-1 were (1) to verify the performance of the TSS equipment; (2) to study the electromagnetic interaction between the tether and the ambient space plasma; (3) to investigate the dynamical forces acting on a tethered satellite. In the first tether deployment, when the satellite was moving excessively from side to side, the deployment was aborted. The second trial of deployment was unreeled to a length of 260 metres.
In 1996, the Tethered Satellite System Reflight (TSS-1R) was carried by using US space shuttle STS-75 successfully. The primary objective of STS-75 was to carry the Tethered Satellite System Reflight (TSS-1R) into orbit and to deploy it spacewards on a conducting tether.
\item{Shuttle Electrodynamic Tether System (SETS)}

	The Shuttle Electrodynamic Tether System (SETS) experiment formed part of the scientific experiments comprising the first flight of the NASA/ASI Tethered-Satellite System flown at an altitude of 300 km and at an orbital inclination of 28.5 degrees in July/August 1992. The SETS experiment was designed to study the electrodynamic behaviour of the Orbiter-Tether-Satellite system, as well as to provide background measurements of the ionospheric environment near the Orbiter. The SETS experiment was able to operate continuously during the mission thereby providing a large data set.
\item{Small Expendable Deployer System (SEDS)}

The Small Expendable Deployer System-1 (SEDS-1) was launched from Cape Canaveral Air Force Station as a Delta/GPS secondary payload in 1993. SEDS-1 was the first successful 20-kilometre space-tether experiment. When 1 km of tether remained, active braking was applied by wrapping the tether around a “barber pole” brake. Finally, the braking system and sensors did not work as predicted, resulting in hard stop/endmass recoil at deployment completion.
The Small Expendable Deployer System-2 (SEDS-2) was launched on the last GPS Block 2 satellite in 1994. The SEDS-2 used feedback braking, which was started early in deployment. This limited the residual swing after deployment to 4 degrees. Mission success was defined as deployment of at least 18 km, plus a residual swing angle of less than 15 degrees. The SEDS-2 had an improved braking system compared to SEDS-1, which was a feedback control system and applied braking force as a function of the measured speed of the unrolling tether. This was to ensure that the satellite stopped flying out just when the whole tether was deployed and to prevent the bounces experienced during the previous mission.
\item{Plasma Motor Generator (PMG)}

In 1996, the Plasma Motor Generator (PMG) was launched by NASA. This was an electrodynamic tether, which could assess the effectiveness of using hollow cathode assemblies to deploy an ionised gas, and to “ground” electrical currents by discharging the energy to space. An early experiment used a 500 metre conducting tether. When the tether was fully deployed during this test, it generated a potential of 3,500 volts. This conducting single-line tether was severed after five hours of deployment. It was believed that the failure was caused by an electric arc generated by the conductive tether’s movement through the Earth’s magnetic field. The PMG flight demonstration proved the ability of the proposed Space Station plasma grounding techniques in maintaining the electrostatic potential between the Space Station and the surrounding plasma medium, the ability to use electrostatic tethers to provide thrust to offset drag in LEO space systems and the use of direct magnetic (nonrocket) propulsion for orbital maneuvering.
\item{Tether Physics and Survivability Experiment (TiPS)}

The Tether Physics and Survivability Experiment (TiPS) was deployed on June 20, 1996 at an altitude of 1,022 kilometres as a project of the US Naval Research Laboratory. The satellite was a tether physics experiment consisting of two end masses connected by a 4 km nonconducting tether. This experiment was designed to increase knowledge about gravity-gradient tether dynamics and the survivability of tethers in space.

\item{The Young Engineers's Satellite (YES)}

The first Young Engineers’ Satellite (YES-1) programme was completed on November 3, 1997. It was designed to operate with a 35 km tether deployment, but the mission was cancelled before the flight when the launch authority changed the nominal Ariane orbit. In the new orbit configuration, a deployed 35 km tether would have constituted a hazard to the satellites in LEO.
The second Young Engineers Satellite (YES2) was launched on September 14, 2007. It was a technology demonstration project designed to test and produce data for the “Space Mail” concept, wherein a tether was used to return material from space to Earth, instead of by conventional chemical
propulsion. YES2 aimed to demonstrate a tether-assisted reentry concept, whereby the payload would be returned to Earth using momentum provided from a swinging tether. Deployment was intended to take place in two phases: (1) deployment of 3.5 km of tether to the local vertical and hold and (2) deployment to 30 km for a swinging cut. The measured altitude gain of the Fonton-M3 corresponded with what simulations showed would have happened if 31.7 km of tether were extended, another strong indication that the YES- 2 tether had in fact been fully depolyed.
The YES-2 mission was very nearly a complete success because of the following: (1) the entire record-breaking length of tether has been deployed; (2) Fotino rocket seemed to have been deorbited by using momentum exchange; (3) plentiful data has been gathered on tether deployment, dynamics and deorbiting, which may lead to an operational way of returning capsules without any form of propulsion

\item{The Advanced Tether Experiment (ATEx)}

The Advanced Tether Experiment (ATEx) was launched into orbit aboard the National Reconnaissance Office (NRO) sponsored by Space Technology Experiment spacecraft (STEX) on October 3, 1998. ATEx was intended to demonstrate the deployment and survivability of a novel tether design, as well as being used for controlled libration maneuvers. On January 16, 1999, after a deployment of only 22m of tether, ATEx was jettisoned from STEX due to an out-of-limit condition sent by the experiment’s tether angle sensor. The ATEx lower end mass was jettisoned from the host spacecraft and the tethered upper and lower end masses freely orbited the Earth in a demonstration of long-term tether survivability. 
\item{The PICOSAT mission}

The PICOSAT mission was launched on September 30, 2001. It was a real-time tracking satellite of the miniaturized picosatellite satellite series. The name “PICO” combined the first letters of all the four of its experiments, which were the Polymer Battery Experiment (PBEX), the Ionospheric Occultation Experiment (IOX), the Coherent Electromagnetic Radio Tomography (CERTO), and the On Orbit Mission Control (OOMC). A pair of 0.25kg MEMS picosatellites with an intersatellite communications experiment were included in this mission and were connected by a 30 metre tether.
\item{Propulsive Small Expendable Deployer System (ProSEDS)--\em{Cancelled}}
	
The Propulsive Small Expendable Deployer System (ProSEDS) was a NASA space tether propulsion experiment intended to be a follow up to SEDS. It was originally intended to be flown along with a launch of a Global Positioning System (GPS) satellite in the spring of 2003, but was cancelled at the last moment, due to concerns that the tether might collide with the international space station.

\item{Multi-Application Survivable Tether (MAST)}

The Multi-Application Survivable Tether (MAST) experiment was launched into LEO on April 17, 2007, in which the 1 km multistrand interconnected tether (Hoytether) was intended to test and prove the long-term survivability of tethers in space, but the tether failed to deploy. 
\item{Tether Technologies Rocket Experiment (T-REX)}

Tether Technologies Rocket Experiment (T-REX) mission was launched August 31, 2010, on sounding rocket S-520-25, reaching a maximum altitude of 309 km. T-Rex was developed by the Kanagawa Institute of Technology/Nihon University, which led an international team to test a new type of electrodynamic tether (EDT) that may lead to a generation of propellantless propulsion systems for LEO spacecraft. The tether was 300 metres long and deployed as planned, a video of deployment was transmitted to the ground. Tether deployment was verified successfully, as was the fast ignition of a hollow cathode in the space environment.
\end{enumerate} 

\section{CubeSat missions with space tether}
1.	Micro Electro-Mechanical Systems based PicoSat Inspector (MEPSI)
The MEPSI series (Micro Electro-Mechanical Systems based PicoSat Inspector) was a pair of tethered picosatellites, based on the CubeSat design, launched by a custom deployer aboard the STS-113 Endeavour mission on December 2, 2002. The two spacecrafts were cubic in shape, of mass 1 kg each, and were connected via a 15.2m tether in order to facilitate detection and tracking via ground-based radar.


2.	The Space Tethered Autonomous Robotic Satellite (STARS)
On January 23, 2009, a space tether mission called“The Space Tethered Autonomous Robotic Satellite (STARS),” developed by Kagawa Satellite Development Project at Kagawa University, was launched as a secondary payload aboard H-IIA flight. The satellite was named KUKAI after launch and based on a “cubesat” platform like MAST, and consisted of two subsatellites, “Ku” and “Kai,” which are linked by a 5-meter tether. The separation of the rocket and satellite and the transfer into the planned orbit were successful, but the tether—only deployed to a length of several centimeters because of the launch lock trouble of the tether reel mechanism .

3.	Tether Electrodynamic Propulsion CubeSat Experiment (TEPCE)
Tether Electrodynamic Propulsion CubeSat Experiment (TEPCE) mission, planned by Naval Research Laboratory, is an electrodynamic tether experiment based on a “triple cubesat” configuration. This experiment is currently planned for launch as a secondary payload in September 2013. Two nearly identical endmasses with a stacer spring between them are used in TEPCE, which separate the endmass and start deployment of a 1 km long braided-tape conducting tether. TEPCE will use a passive braking to reduce speed and hence recoil at the end of electrodynamic current in either direction.
The main purpose of this mission is to raise or lower the orbit by several kilometres per day, to change libration state, to change orbit plane, and to actively maneuver.


\section{Design and construction}

\subsection{Space tether deployment/retrieval system}

\subsection{Plasma contactor}

\section{Space tether dynamical modelling}



\newpage

%\chapter{Preliminaries and Related Work}\label{sec-preliminaries}

In this chapter, we first give a brief introduction on the GPU, and then review the related work on GPGPU as well as on MapReduce.


\section{Graphics Processing Units (GPUs)}

The GPU is an integral component of
modern computers, ranging from handheld devices to high-end servers.
GPUs are originally designed for gaming applications with fixed
hardware pipelines for rendering. Due to the high computation power
and rapidly improving programmability, they have recently become a
powerful co-processor for general purpose computing~\cite{Ailamaki2006}.
%For example, NVIDIA Tesla GPU series have been
%adopted to high performance computers and clusters~\cite{Tesla}. For
%more details on the GPU and its programming techniques, we refer the
%reader to a recent book edited by Nguyen~\cite{GPUGems}.


\begin{figure}[ht]
\centering
\includegraphics[width=0.65\textwidth]{figure/gpuarch.eps}
\caption{The many-core architecture model for GPUs}
\label{fig:manycore}
\end{figure}



%The major vendors such as NVIDIA and AMD provide similar programmable hardware pipelines, and develop similar programming frameworks.
As shown in Figure \ref{fig:manycore}, we model the GPU as a many-core processor, which
contains a number of SIMD multiprocessors.
Such a many-core model is common to both AMD and NVIDIA GPUs.
On the GPU board, there is a GRAM device memory.
The device memory has both a high bandwidth and a high access latency.
For example, the NVIDIA GTX280 GPU has an access latency of 400 to 600 cycles, and
the peak memory bandwidth between the device memory and the multiprocessors
is around 140 GB/second.

Both NVIDIA CUDA and AMD Brook+ expose a parallel programming model, which does not require programmers to have knowledge of the
graphics rendering pipeline. In this model, the system consists of a
{\em host} (a CPU), and one or more {\em devices} (GPUs).
GPUs are abstracted as massively data-parallel co-processors.
CUDA and Brook+ programmers write code using C/C++ syntax with extended keywords for kernel functions,
which are GPU programs to be executed on {\em devices}.

Programming frameworks such as CUDA and Brook+ greatly improve
the programmability of the GPU.
However, it is still a challenging task of developing efficient GPU programs for complex applications, such as those with MapReduce, because GPUs have a special-purpose co-processor architecture and are vendor-specific on the programming frameworks for complex applications.
\red{Although the newly introduced OpenCL~\cite{OPENCL} is an industry standard further hiding hardware details from users, Mars is at a higher level of abstraction.} OpenCL is a general-purpose programming language, with which Mars or other MapReduce frameworks can be developed.

\section{GPGPU}




GPGPU, or General Purpose Computation on GPUs, has recently emerged
in various applications, such as linear algebra~\cite{Jiang2005,Volkov2008}, embedded system design~\cite{Feng2007}, bioinformatics~\cite{Charalambous2005}, databases~\cite{Govindaraju2006,Govindaraju2004,He2008a}, machine learning~\cite{Chu2006}, and
distributed computing projects including Folding@home and Seti@home.
Recently, several GPGPU languages including AMD Brook+~\cite{BROOKPLUS} (extended from Brook~\cite{Buck2004}) and NVIDIA
CUDA~\cite{CUDA} have been proposed by GPU vendors. They usually
expose a general-purpose, massively multi-threaded computing
architecture and provide a programming environment similar to C/C++.
High-level programming frameworks such as Accelerator~\cite{Tarditi2006} and RapidMind~\cite{McCool2006} are also
developed to better facilitate GPGPU programming. These
programming frameworks require programmers to have knowledge of specific
programming models such as the stream programming model in Brook+~\cite{Buck2004}, or even more, knowledge of the GPU hardware
details. By contrast, we propose to develop a MapReduce framework accelerated with
GPUs to ease the development of a more complex class of data
processing tasks. It provides a uniform MapReduce interface no matter whether it runs on the GPU, on the CPU, or both.

We now briefly survey recent  work that developed GPGPU primitives as building blocks for various applications, in particular, those not covered in the survey by Owens et al~\cite{Owens2007}.
Sengupta et al.~\cite{Sengupta2007} proposed the segmented scan primitive.
He et al.~\cite{He2007} proposed a multi-pass scheme to optimize the scatter and the gather operations.
He et al.~\cite{He2008a} further developed a small set of primitives such as prefix sum and split for relational databases.
Additionally, CUDPP~\cite{CUDPP}, a CUDA library of data parallel primitives, was released for GPGPU computing.
These GPU-based primitives reduce the complexity of GPU programming.
However, even with the primitives, programmers need to write complex GPU code for data processing tasks.
By contrast, our work further simplifies GPU programming for MapReduce programmers by providing them with a higher level and more familiar interface than the primitives.

\red{This thesis focuses  on  accelerating  MapReduce  on  the  GPU,  and  provides  a  GPU-based MapReduce framework to developers. As in the original MapReduce,  it  is  up  to  developers'  choice  whether to  use  MapReduce  or  not  according  to the workload's computational characteristics. Recent studies \cite{Kerr2010}  have used data analysis techniques to categorize the computational characteristics of different workloads on the GPU. These techniques are helpful for developers to determine whether their workloads are suitable for Mars in
specific and the GPU in general.}

%Our previous study on Mars~\cite{He2008} implemented the MapReuduce framework on CUDA-enabled GPUs. This work extends the previous work in two major aspects. First, we extend the CUDA-only Mars to another large group of GPUs, so that it can run on both NVIDIA and AMD GPUs. Second, we use GPU-only Mars as a component to work with CPU-based Mars on a single machine, as well as with Hadoop in a distributed environment.

\section{MapReduce}
The MapReduce framework~\cite{Dean2008} is based on two primitives, Map and Reduce, from functional programming.
The general form is as follows:
\begin{quote}
\tab \bf{Map:} $(k_1, v_1) \rightarrow list(k_2, v_2)$.

\tab \bf{Reduce:} $(k_2, list(v_2)) \rightarrow list(k_3, v_3)$.
\end{quote}



The Map function takes an input key/value pair $(k_1, v_1)$ and outputs a list of intermediate key/value pairs $(k_2, v_2)$.
The Reduce function takes all values associated with the same key and produces a list of key/value pairs.
Programmers implement the application logic inside the Map function and the Reduce function.
The MapReduce runtime manages the parallel execution of these two functions.



The following pseudo code illustrates a program written using MapReduce.
This program counts the number of occurrences of each word in a collection of documents~\cite{Dean2008}.
In this program, {\em Map} and {\em Reduce} are implemented using two system-provided APIs, {\em EmitIntermediate} and {\em Emit}, respectively.

\begin{quote}
{\bf Map}(void *{\em doc}) \{ \\
1: {\bf for} each word {\em w} in {\em doc} \\
2: \hspace{3mm} {\bf EmitIntermediate}({\em w}, {\em 1}); // count each word once \\
\} \\
{\bf Reduce}(void *{\em word}, Iterator {\em values}) \{ \\
1: int {\em result} = 0; \\
2: {\bf for} each {\em v} in {\em values} \\
3: \hspace{3mm} {\em result} += {\em v}; \\
4: {\bf Emit}({\em word}, {\em result}); // output {\em word} and its count \\
\}
\end{quote}

%\begin{quote}
%\tab {\bf Map}(void *{\em doc}) \{ \\
%\tab 1: \tab {\bf for} each word {\em w} in {\em doc} \\
%\tab 2: \tab \tab {\bf EmitIntermediate}({\em w}, {\em 1}); // count each word once \\
%\tab \} \\
%\tab {\bf Reduce}(void *{\em word}, Iterator {\em values}) \{ \\
%\tab 1: \tab int {\em result} = 0; \\
%\tab 2: \tab {\bf for} each {\em v} in {\em values} \\
%\tab 3: \tab \tab {\em result} += {\em v}; \\
%\tab 4: \tab {\bf Emit}({\em word}, {\em result}); // output {\em word} and its count \\
%\tab \}
%\end{quote}

There have been several MapReduce implementations since MapReduce was proposed~\cite{Dean2008}.
Hadoop~\cite{HADOOP} is an open-source MapReduce implementation on clusters.
Based on Hadoop, Yang et al.~\cite{Yang2007} added the merge operation to MapReduce for the ease of relational databases operations.
Phoenix~\cite{Ranger2007} is an efficient MapReduce runtime system on multi-core CPUs.
Kruijf et al.~\cite{Kruijf2007} developed MapReduce on the Cell BE.
Yeung et al.~\cite{Yeung2008} implemented an FPGA-based MapReduce system.

\red{Let us briefly introduce the implementation of Phoenix~\cite{Ranger2007}. A key component in Phoenix is a scheduler, for buffer management and task distribution. The scheduler starts the Map stage by evenly dividing the input buffer into small chunks, and assigns the chunks to map workers dynamically. Each map worker runs in a CPU thread. The Reduce stage does not start until all Map tasks are done. The scheduler groups the intermediate output from the Map stage by key, and a Reduce worker processes values associated with the same key. Reduce tasks are assigned to workers dynamically. Each reduce worker maintains a static array for outputting results, and sorts this static array using insertion sort. Finally, the scheduler merges all output arrays of reduce workers into a single one. Because the output data size is not known in advance, the scheduler first allocates buffers with a default small size, and then resizes the buffer as needed.}


%Our previous work on Mars implemented MapReduce on CUDA-enabled GPUs~\cite{He2008}. 
Catanzaro et al.~\cite{Catanzaro2008}, developed another MapReduce system on the GPU, but it required programmers to be aware of GPU hardware details, such as thread configuration and memory hierarchy.
Finally, the Merge framework~\cite{Linderman2008}, focused on dynamically scheduling MapReduce tasks among multiple processors, dedicated to Intel products.
By contrast, Mars hides hardware details from programmers, and works on heterogeneous GPUs, a combination of CPU and GPU on a single machine, as well as a distributed system of multiple machines.

\newpage

%\chapter{Design of Mars} \label{sec-design}

In this chapter, we present our design for Mars, with emphasis on the GPU-based component.
Our design is guided by the following three goals.

\begin{enumerate}
  \item {\em Programmability}. User code size reduction encourages programmers to use the GPU for their tasks.
  \item {\em Flexibility}. The design should be applicable to various multi-/many-core processors, e.g., multi-core CPUs and AMD GPUs, and should be as expressive as the underlying runtime, e.g., NVIDIA CUDA, AMD Brook+ or pthreads, so that the system will work for a wide range of hardware and applications.
  \item {\em High Performance}. The overall performance should be accelerated by the GPU effectively.
\end{enumerate}

\section{Overview}
\red{By examining the Phoenix design, we see that there are three potential sources of overhead.  First, the tight-coupling of the Map and Reduce stages makes every application go through both stages, no matter whether they need both stages or not. Second, a dynamic thread scheduler for task assignment heavily relies on locking to implement synchronization. Third, each reduce worker may require frequent data movement for sorting the static output array, and the data movement can become a bottleneck for the overall performance. The latest paper about Phoenix also points out this problem~\cite{Yoo2009}.}


In the Mars design, we decide to separate a MapReduce workflow into three loosely coupled stages -
{\em Map}, {\em Group} and {\em Reduce}. The {\em Group} stage is designed to group {\em Map} output by key, which is the format for {\em Reduce} input. Our observation is that some applications need only the
{\em Map} stage, some need both {\em Map} and {\em Group}, and some need all of the three stages.
\red{The Group stage is the same as running Reduce with the identity function in the original MapReduce system \cite{Dean2008}.
Our purpose of providing an explicit Group stage is to allow a MapReduce application with high flexibility to customize its workflow,
and to avoid the overhead of entering unnecessary stages. }
No matter what configuration of the three stages is for an application, the MapReduce interface
of Mars is unchanged - users write Map and/or Reduce functions when necessary.

Moreover, we decide to use a lock-free scheme for synchronization
and to perform in-advance buffer allocation. One reason is to
avoid heavy overheads of locking and buffer reallocation. The
other reason is that, current GPUs do not support locking
or in-flight buffer reallocation. In our design, we statically
distributes tasks to a massive number of GPU threads, so that we
can fully utilize the parallelism of the GPU. We adopt a two-phase,
lock-free scheme for result output. The basic idea is that, in the
first phase, we calculate histograms on the size of output results for each
thread, followed by a prefix sum operation on the histograms,
so that we obtain both the exact output buffer size and the deterministic
write position for each thread; in the second phase, we
perform the actual computation and output. We will detail this strategy in
Section~\ref{sec-lockfree}.

\section{Data Structure} \label{sec-datastructure}
Data structures in Mars affect the workflow, memory access patterns, and the expressiveness of the system.

Since the GPU does not support dynamic memory allocation on the
device memory during the execution of the GPU code, this limitation rules out
 dynamic data structures such as queues and linked lists, as used in
other MapReduce implementations. Instead, we use plain arrays as the
main data structure in Mars. The {\em Map} stage takes input records in
the key/value form, and outputs intermediate result records, which
are in turn the input of the {\em Group} stage. The output of the {\em Group}
stage is the input of the {\em Reduce} stage, and {\em Reduce} produces final output
records. Each of the three sets - the input records, the intermediate records and the
output records  - is stored in three arrays, i.e., the
key array, the value array and the directory index array. The
directory index consists of an entry of $<$key offset, key size,
value offset, value size$>$ for each key/value pair. Given a
directory index entry, we fetch the key or the value at the
corresponding offset in the key array or the value array.

Variable-sized types, such as strings, are supported with the directory index, since current GPUs have no such build-in types yet. If
two key/value pairs need to be swapped, we swap their corresponding
entries in the directory index without modifying the key and the
value arrays.

Some applications perform chained MapReduce procedures,
where the output of one MapReduce procedure is the input of
another one. Since the sets of
input records, intermediate records and output records  are all
in the three-array structure uniformly, chained MapReduce is
supported gracefully in Mars.

\section{Mars Workflow}
Figure \ref{fig:MarsWorkflow} illustrates the workflow of Mars, assuming the data resides in the disk at the beginning.
The Mars scheduler runs on the CPU, and schedules tasks to the GPU.
Mars has three stages, {\em Map}, {\em Group}, and {\em Reduce}.

\begin{figure}[ht]
  \centering
  \includegraphics[width=0.60\textwidth]{figure/Mars_workflow.eps}
  \caption{The workflow of Mars on the GPU. }\label{fig:MarsWorkflow}
\end{figure}

Before the {\em Map} stage, Mars preprocesses on the CPU the input data from disk,
transforming the input data to key/value pairs (input records) in main memory.
After that, it transfers input records from the main memory to the GPU device memory.

In the {\em Map} stage, {\em Map Split} dispatches
input records to GPU threads such that the workload for all threads is even.
Each thread executes the user-defined MapCount function to compute a local histogram on
the number and the total size of intermediate records that {\em Map} will output.
Then, the runtime performs a GPU-based {\em Prefix Sum} on the local histograms to
obtain the output size and the write position for each thread.
Finally, after the CPU allocates the output buffer in the device memory, each GPU
thread executes the user-defined Map function and outputs results.  Since the write
position for each thread is pre-computed and has no conflict with any other threads,
there will be no write conflict between concurrent threads. This lock-free scheme of MapCount,
Prefix Sum, and Map is adapted from our previous
work~\cite{He2008a}.

\red{In the {\em Group} stage, both sort-based and hash-based approaches are available for grouping records by key. However, we adopt the sort-based, because some applications require to sort all output records, and the hash-based approach has to perform additional sort within each hash bucket. }


In the {\em Reduce} stage, {\em Reduce Split} dispatches each group of records
with the same key to a GPU thread.
However, it may cause load imbalance between threads, since the number of records of different groups may vary widely.
We adopt a skew handling scheme to alleviate the load imbalance problem (Section~\ref{sec-reduce}).
The {\em Reduce} stage then works in a lock-free scheme, similar to that in {\em Map}, to obtain the result size and the write location for each thread.
Finally, all Reduce workers output results to a single buffer.

Because these three stages are loosely coupled and not every application requires all
stages, Mars allows users to customize the following workflows in their applications:
\begin{itemize}
\item MAP\_ONLY. Mars executes the {\em Map} stage only, and does not execute the {\em Group} or {\em Reduce} stage.
\item MAP\_GROUP. Mars executes the {\em Map} and {\em Group} stages, and does not execute the {\em Reduce} stage.
\item MAP\_GROUP\_REDUCE.  Mars executes all three stages -- {\em Map}, {\em Group}, and {\em Reduce} stages.
\end{itemize}

Because usually applications need a Map to transform input records, and a Group to
prepare for the intermediate records to feed to Reduce, we exclude the other
workflow configurations that skip either Map or Group in the presence of Reduce.

\section{Lock-free Scheme} \label{sec-lockfree}

With the array structure, we allocate the space on the device memory
for the input data as well as for the result output before executing
the GPU program. However, the sizes of the output from the {\em Map} and
the {\em Reduce} stages are unknown. Moreover, write conflicts occur when
multiple threads write results to the shared output array. To
address these two problems, we adopt a previous lock-free output
scheme for relational joins~\cite{He2008a}. Since the output scheme
for the {\em Map} stage is similar to that for the {\em Reduce} stage, we
present the scheme for the {\em Map} stage only.

First, each MapCount invocation on a thread outputs three counts, i.e.,
the number of intermediate results, the total size of intermediate keys (in bytes)
and the total size of intermediate values (in bytes).
Based on intermediate key sizes (or value sizes),
Mars computes a prefix sum on these sizes and produces an array of write locations.
A write location is the start location in the output array for a map task to write.
Based on the number of intermediate results,
Mars computes a prefix sum and produces an array of start locations in the output
directory index.
Through these prefix sums, we also know the sizes of the arrays for the intermediate
results.
Finally, Mars allocates arrays in the device memory with the exact sizes for
storing the intermediate results.

Second, each Map invocation on a thread outputs the intermediate key/value pairs to
the output array.
Since each Map has its deterministic and non-overlapping positions to write to,
the write conflicts are avoided.


The lock-free scheme is suitable for the massive thread parallelism on the GPU, even
though it performs a MapCount in addition to a Map. The overhead of executing MapCount
is application dependent, and is usually small. For example, this overhead is negligible
in the matrix multiplication in our study, since MapCount simply emits the size without
performing the actual multiplication. In addition, the code for MapCount function is also application dependent, while in most cases, programmers write one statement to emits output data sizes. 

\red{
\section{Rapid Group} \label{sec-rapid}
The Group stage requires to sort intermediate records.
However, we observe that some applications inherently have their intermediate records grouped after the Map phase, and each group has the same number of records.
For example, [A,A,A,B,B,B,C,C,C] shows three groups with A, B, and C as the key respectively, and each group is with the same size 3.
For such applications, Mars provides a configuration parameter for users to indicate whether the intermediate data is already grouped. The runtime automatically skips the time consuming sorting, and then dispatches each group of intermediate records with the same group size to Reduce workers.
We name this strategy as ``Rapid Group".
}

\section{Skew handling} \label{sec-reduce}

We design a skew handling scheme to distribute workloads evenly across reduce workers, where the user-defined Reduce operation is commutative and associative.
This scheme iteratively performs the {\em Reduce} stage in the following two steps.  First, we divide the data into $M$ equal-sized chunks.  Second, we perform a reduction on each chunk. In this step, each of the $M$ threads applies the reduce function on groups of records in a single chunk.  Note, in each iteration, we perform reduction on the intermediate results with the same keys only.

%This scheme starts within the {\em Group} stage, immediately after the sorting, and performs in the following three steps.  First, we divide the data into $M$ equal-sized chunks. Second, we perform a reduction on each chunk.  In this step, each of the $M$ threads applies the reduce function on groups of records in a single chunk, called {\em local reduction}.  Finally, we group the reduced intermediate results and pass them to the {\em Reduce} stage.  This skew handling scheme makes sure that {\em local reduction} in {\em Group} stage are load-balanced.  Additionally, the {\em Reduce} stage processes compact groups of records, so that it alleviates the bottleneck caused by the largest group.

\section{Mars APIs}
Mars provides a small set of APIs. Similar to the existing MapReduce
frameworks, Mars has two kinds of APIs, the user-implemented APIs,
which the users should implement by themselves, and the
system-provided APIs, which the users can use as library calls.
%We refer the readers to our conference paper~\cite{He2008} for more details about Mars APIs.
The definitions of these APIs are in Table \ref{tb:marsapi}.

\doublerulesep 0.1pt
\begin{table}[htb]
  \centering
 \linespread{1.7}{ {\footnotesize
  \caption{Mars APIs}\label{tb:marsapi}
\vspace{2em}
  \begin{tabular}{p{5cm}p{7.5cm}p{3.5cm}}
  \hline
\noalign{\smallskip}
   \textbf{Function Name} & \textbf{Description} & \textbf{Function Type}\\
\noalign{\smallskip}
  \hline
   MAP\_COUNT & It calculates the output buffer size of MAP. & User-implemented \\
   MAP & The map function. & User-implemented \\
   REDUCE\_COUNT & It calculates the output buffer size of REDUCE. & User-implemented \\
   REDUCE & The reduce function. & User-implemented \\
   EMIT\_INTERMEDIATE\_COUNT & It emits the key size and the value size in MAP\_COUNT. & System-provided \\
   EMIT\_INTERMEDIATE & It emits the key and the value in MAP. & System-provided \\
   EMIT\_COUNT & It emits the key size and the value size in REDUCE\_COUNT. & System-provided \\
   EMIT & It emits the key and the value in REDUCE. & System-provided \\
\noalign{\smallskip}
  \hline
  \end{tabular}
  }}
\end{table}



\newpage

%\chapter{Single-Machine Implementations}\label{sec-implement}
In this chapter, we present the implementation details of Mars on a
single machine. The current Mars system consists of four modules
(Table \ref{tb:impls}). All these four modules share the common
design of Mars, and provide the same MapReduce interface to the
user. They can run on different hardware platforms: MarsCUDA on
an NVIDIA GPU, MarsBrook on an AMD GPU, MarsCPU on a multi-core CPU,
and the GPU/CPU co-processing module on both the CPU and the GPU
through combining the aforementioned modules.  Different modules in
Mars allow programmers to take advantage of different processors on
a single machine. Because our machines cannot host a multi-GPU configuration due to limited extension slots, we have not explored multi-GPU co-processing.

\begin{table}[htb]
  \centering
 \linespread{1.7}{ {\footnotesize
  \caption{Modules in the Mars system}\label{tb:impls}
\vspace{2em}
  \begin{tabular}{ccc}
  \hline
\noalign{\smallskip}
  \textbf{Implementation} & \textbf{Software platform} & \textbf{Hardware platform}\\
\noalign{\smallskip}
  \hline
  MarsCUDA & NVIDIA CUDA & an NVIDIA GPU \\
  MarsBrook & AMD Brook+ & an AMD GPU \\
  MarsCPU & pthreads & a multi-core CPU \\
  GPU/CPU co-processing & CUDA/Brook+ and pthreads &  NVIDIA/AMD GPUs and multi-core CPUs\\
\noalign{\smallskip}
  \hline
  \end{tabular}
  }}
\end{table}

\section{MarsCUDA}

We implemented MarsCUDA using NVIDIA CUDA. We used the GPU
Prefix Sum routine from CUDPP~\cite{CUDPP} to implement the lock-free scheme, and
the GPU Bitonic Sort routine for the {\em Group} phase.
CUDA exposes sufficient hardware details of NVIDIA GPUs, so that we can
apply some optimizations in MarsCUDA runtime.
\\\\
{\em 1. Memory access}
\\
{\bf Coalesced access.} We utilize the NVIDIA GPU feature of
coalesced access to improve the memory performance. In CUDA,
simultaneous device memory accesses by threads in a half-warp (warp is an NVIDIA term for a group of 32 threads for scheduling) can be
coalesced into a single memory transaction, which significantly
reduces the number of device memory accesses.
We implement the access to the directory index arrays as coalesced.

\red{{\bf Local  memory.} NVIDIA  GPUs  provide  the programmable  on-chip  local  memory  (or \emph{shared memory}~\cite{CUDA2008}), for sharing data among threads running on the same multiprocessor. It is important to fully utilize the local memory to reduce the costly accesses to the GPU memory. In Mars, data sharing or communication only happens in the Group stage. MarsCUDA runtime automatically uses a GPU-based bitonic sort~\cite{He2008} to exploit this memory hierarchy in the Group stage. Mars does not expose the local memory to the user-defined functions in the Map and the Reduce stages. Since local memory is programmer-controlled fast memory, it introduces complexity and needs the effort from the programmer. This is a trade-off between performance and programmability.  Nevertheless, users who are aware of the GPU memory hierarchy and need such data sharing can exploit the local memory in implementing the Map (or Reduce) function.
}
%{\bf Local memory.} NVIDIA GPUs provide programmable on-chip local memory (or shared memory in CUDA term), for sharing data among threads running on the same multiprocessor.  It is critical to well utilize local memory to achieve high GPU memory bandwidth.  In MapReduce framework, data sharing or communication only happens between the Map stage and the Reduce stage, that is, the Group stage in Mars.  MarsCUDA runtime automatically exploits this memory hierarchy in the Group stage, which is a GPU-based bitonic sort.  During the Map (or Reduce) stage, each Map (or Reduce) task is independent, and there is little opportunity for data sharing between the tasks.  Moreover, since local memory must be programmed explicitly, it introduces complexity and needs the effort from the programmer. This is a trade-off between performance and programmability.  Nevertheless, users who are aware of the GPU memory hierarchy and need such data sharing can exploit the local memory in implementing the Map (or Reduce) function.

\red{
{\bf Built-in vector types.}
%Accessing data values in the device memory can be costly, because they are often of different sizes and the accesses are hardly coalesced.  Fortunately, GPUs support built-in vector types \cite{CUDA2008}, including {\em char4} and {\em int4}.
Data accesses in the GPU device memory should be aligned to make sure the correctness and achieve high memory bandwidth.
Fortunately, GPUs support built-in vector types \cite{CUDA2008}, including {\em float4} and {\em int4}.
The alignment requirement is automatically fulfilled for built-in types.
%A read of a built-in vector fetches the entire vector in a single memory request.
In addition, the GPU is able to issue a single load instruction to read data of built-in type, of size up to 16 bytes.
%Compared with reading {\em char} or {\em int}, the number of memory requests in reading {\em char4} or {\em int4} is greatly reduced and the memory performance is improved.
Compared with reading an array one {\em float} or {\em int} at a time, the number of compiler-generated instructions for reading {\em float4} or {\em int4} is greatly reduced and the overall performance is improved.
}

\red{
{\bf Page-locked host memory.} CUDA supports page-locked host memory (a.k.a pinned), which prevents the operating system from paging the locked memory buffer, yielding high transfer bandwidth between the device memory and the host memory \cite{CUDA2008}.
The MarsCUDA runtime utilizes the page-locked host memory mechanism, in order to reduce the data transfer overhead. Our test demonstrated that page-locked memory can double the memory transfer rate through PCI-E bus than pageable memory.}
\\\\
{\em 2. Parallelism}
\\
Since CUDA exposes the thread configuration, we utilize the parallelism by assigning the tasks to
a large number of threads. The thread configuration, i.e., the number of thread blocks and the
number of threads per thread block, is related to both hardware and software factors:
 (1) the hardware configuration such as the number of
multiprocessors and the on-chip computation resources such as the
number of registers on each multiprocessor, and (2) the characteristics of the map and the reduce tasks, e.g., the degree of memory- or computation-intensiveness.

Since the map and the reduce functions are implemented by the
developer, and their costs are unknown to the runtime system, it is
difficult to find the optimal setting for the thread configuration
at run time. CUDA provides an off-line
calculator~\footnote{http://developer.download.nvidia.com/compute/cuda/CUDA\_Occupancy\_calculator.xls}
for computing the multiprocessor occupancy given a CUDA program. For
the program (either the map task or the reduce task), the calculator
takes the number of threads per thread block and the number of
registers used per thread as input, and outputs the occupancy and
the number of active thread blocks per multiprocessor. The number of
registers used per thread is obtained using the NVCC compiler of
CUDA.

With the calculator, we iterate the number of threads per block in
multiples of 32 (the schedule unit size) ranging from 32 to 512 (the
maximum number of threads per thread block), until the occupancy is
higher than a predefined threshold. Thus, we get the number of
threads per thread block and the number of thread blocks. In
practice, we set the occupancy threshold to be 2/3 so that the GPU
is sufficiently busy, and each thread block receives adequate
computation resources.


%We determine the thread configuration according to the warp occupancy on the multiprocessor.
%The thread configuration includes the number of threads per thread group $N_{t}$,
%and the number of thread groups $N_{g}$.
%Multiple factors determine the thread configuration, including
%the hardware configuration such as the number of multiprocessors and
%the on-chip computation resources such as the number of registers on each multiprocessor
%and the usage of on-chip local memory (shared memory).


%Since the map and the reduce functions are implemented by the developer, and their costs are unknown to the runtime system,
%it is difficult to find the optimal setting for the thread configuration at run time.
%Fortunately, CUDA provides an off-line calculator for computing the multiprocessor warp occupancy given a CUDA program,
%and then determine $N_{t}$.
%For the program (either the map task or the reduce task), the calculator takes the number of threads per thread group $N_{t}$,
%the number of registers used per thread and the usage of local memory per block as input,
%and outputs the occupancy, which is the number of active warps per multiprocessor.
%The number of registers used per thread and the usage of local memory per block are obtained using the NVCC compiler of CUDA.
%With the calculator, we iterate the number of threads per group in multiples of 32 (the schedule unit size) ranging from 32 to 512
%(the maximum number of threads per thread group), until the occupancy is higher than a predefined threshold.
%Thus, we get the number of threads per thread group $N_{t}$.
%In practice, we set the occupancy threshold to be 2/3 so that the GPU is sufficiently busy,
%and each thread group receives adequate computation resources.
%
%
%We obtain the number of threads per thread group $N_{t}$ manually,
%and obtain the number of thread groups $N_{g}$ automatically during the runtime.
%For the Map stage, given the number of input records $N_{r}$, we get:
%\begin{center}
%$N_{g} = \lceil \frac{N_{r}}{N_{t} \times num\_rec\_map} \rceil$
%\end{center}
%$num\_rec\_map$ is the parameter configured by users, meaning the number of records processed per map task.
%It is similar for the Reduce stage, so we skip that.

\section{MarsBrook} \label{sec-brook}
We implement MarsBrook on AMD GPUs using the stream programming
model Brook+ \cite{BROOKPLUS}.
Due to the limitation of Brook+, MarsBrook is less advanced than MarsCUDA in both expressivity and performance.
Nevertheless, as programming support of Brook+ improves, MarsBrook can demonstrate a higher flexibility and performance.

MarsBrook requires users to specify the data types of keys and values statically, and each record is of a fixed size.
Type conversion is not allowed in Brook+. Unlike CUDA, Brook+ does not allow the developer to access data in GPU memory by arbitrary address.
Instead, data in the GPU memory is accessed using a {\em stream}, which is essentially a sequentially-accessed array of fixed-sized elements. Random access in a stream is achieved by providing another predefined stream, consisting of indexes of target elements to access.
\red{Although the Mars APIs are the same on CUDA and on Brook+, as listed in Table \ref{tb:marsapi},
using Mars on CUDA is more flexible than using that on Brook+ when the Mars user develops a user-defined function.
}

Moreover, MarsBrook has relatively limited room for performance optimization.
The reason is that Brook+ does not expose detailed hardware
features, e.g., fast on-chip local memory, coalesced memory access,
or GPU thread configuration.



%While the implementation of MarsBrook shows the general design of
%Mars is also feasible to AMD GPUs, other than NVIDIA GPUs.

\section{MarsCPU}
We implement MarsCPU using the pthreads library on linux for multi-threading.
Instead of adopting lock-based task scheduling as in Phoenix,
MarsCPU inherits the lock-free design of GPU-based Mars, which we expect to scale to hundreds of cores for future many-core CPUs.
MarsCPU deploys CPU threads to perform Map and Reduce tasks.
If there are $N$ Map (or Reduce) tasks, and $T$ CPU threads, where $N$ is usually much larger than $T$, then a thread processes $\lceil N/T \rceil$ tasks.
We implement a CPU multi-threaded parallel mergesort for the Group stage.

\section{GPU/CPU co-processing} \label{sec-coprocessing}

%Since both the CPU and the GPU are integrated components on a
%commodity machine, we develop MapReduce with co-processing on these
%two kinds of processors to fully utilize their computation power.

The workflow of GPU/CPU co-processing is shown in Figure \ref{fig:Mars+}.
There are also mainly three stages, {\em Map}, {\em Group} and {\em Reduce}.
In the {\em Map} stage,
the scheduler divides the input data into multiple chunks. The
number of chunks is equal to the total number of CPUs and GPUs in
the machine. The chunk sizes are determined based on the performance
comparison between the CPU and the GPU. Suppose the speedup of the
GPU worker over the CPU worker is $S$, where the {\em speedup} is
defined to be the ratio of the execution time on the CPU to that on
the GPU for the same amount of input data. Given the total input
size of $I$ bytes, we assign data chunks of $\frac{SI}{1+S}$ and
$\frac{I}{1+S}$ bytes to the GPU and the CPU workers, respectively.
\red{The speedup $S$ can be obtained by either calibration or predictive model \cite{Kerr2010}.}

When a processor finishes a {\em Map} task, it performs a local {\em Group} on intermediate results.
The runtime merges all intermediate results. When all the processors finish their
tasks, the {\em Map} stage ends.

The {\em Reduce} stage takes the intermediate results from the {\em Group} stage
as input. Similar to the {\em Map} stage, the co-processing scheduler statically assigns the
data chunks to the processors. When all the processors finish their
tasks, the runtime merges all local results.

%Please note that, workload dispatching between the GPU worker and the CPU worker by speedup works well only under two conditions.
\red{
Mars dispatches workload between the GPU worker and the CPU worker only if the following conditions are satisfied.
First, the Map and Reduce stages take up high proportion of the entire running time on the CPU worker. If components other than the Map and Reduce stages contribute to a large portion of running time, the GPU worker is not able to make large performance acceleration. Second, the GPU worker and the CPU worker have comparable performance. The benefit of using the CPU worker diminishes, as the speedup of the GPU worker over the CPU worker becomes higher. 
}
\begin{figure}[h]
  \centering
  \includegraphics[width=1.00\textwidth]{figure/Mars+.eps} 
  \caption{The workflow of GPU/CPU co-processing. }\label{fig:Mars+}
\end{figure}

With the GPU/CPU co-processing module, Mars can harness the
computation power of NVIDIA GPUs, AMD GPUs, and multi-core CPUs on
the same machine, by integrating MarsCUDA, MarsBrook, and MarsCPU
modules as components.

%\chapter{Multi-machine Implementations}\label{sec-beyond}

In this chapter, we present the integration of Mars into a CPU-based distributed MapReduce system, specifically Hadoop in our implementation. 
This integration benefits from both worlds: Hadoop utilizes CPUs on multiple machines and provides fault-tolerance and other features of a distributed system; 
Mars utilizes the GPU to accelerate local computation. We denote Mars-enabled Hadoop as MarsHadoop. 

We use the {\em Hadoop Streaming}
technology~\footnote{http://hadoop.apache.org/common/docs/r0.15.2/streaming.html}
to integrate Mars into Hadoop. {\em Hadoop Streaming} enables the
developers to use their own custom Map or Reduce implementation in
Hadoop. In our implementation, we use the Mars executable to read
the input from {\em stdin} and to emit the output to {\em stdout}.
Thus, the Map and the Reduce tasks can be performed on the GPU, and
other tasks such as task scheduling and failure handling are
performed by Hadoop. Finally, since current GPUs do not support
multi-tasking, we configure MarsHadoop to run GPU-based tasks sequentially on one GPU. 


Figure \ref{fig:hadoop} illustrates the workflow of MarsHadoop. 
A Map Worker/Reduce Worker in MarsHadoop is the same as a Map Worker/Reduce Worker shown in Figure \ref{fig:Mars+}; 
in other words, it can be from MarsCUDA, MarsBrook, or MarsCPU, depending on the underlying processor. 
In the configuration of Figure \ref{fig:hadoop}, Node 1 simultaneously runs two Map Workers, on a GPU and a CPU respectively. 

\begin{figure}[ht]
  \centering
  \includegraphics[width=0.70\textwidth]{figure/hadoop.eps} 
  \caption{MarsHadoop. Some Map and Reduce tasks are performed on the
  GPU, and others are on the CPU. }\label{fig:hadoop}
\end{figure}

%\chapter{Experimental Evaluation}\label{sec-eval}
In this chapter, we evaluate Mars on a single machine using a micro-benchmark of six applications in comparison with their CPU-based counterparts and native GPU-based implementations.
\red{We also evaluate the performance of MarsHadoop on two connected machines.}


\section{Experimental Setup}
Our experiments were performed on three PCs, A, B and
 C. Table \ref{tb:machines} shows their hardware configuration.
Both PCs A and B run 32-bit CentOS 5.1 Linux with kernel
2.6.18, NVIDIA CUDA 2.2, and the GPU driver 185.18.14. PC C runs
32-bit Windows XP Pro SP3, with Brook+ 1.01.0 beta, and the GPU
driver 8.561. All hard drives on these PCs are SATA magnetic hard
disks with 7200 rpm. On all PCs, the main memory and the device
memory are connected by PCI-E bus with a theoretical bandwidth of 4
GB/sec.

\doublerulesep 0.1pt
\begin{table}[htb]
  \centering
 \linespread{1.7}{ {\footnotesize
  \caption{Machine configurations}\label{tb:machines}
\vspace{2em}
  \begin{tabular}{p{4.5cm}p{3.5cm}p{3.5cm}p{3.5cm}}
  \hline
\noalign{\smallskip}
    \textbf{Machine} & \textbf{PC A}&\textbf{PC B}&\textbf{PC C}\\
\noalign{\smallskip}
  \hline
    GPU&NVIDIA GTX280& NVIDIA 8800GTX & ATI Radeon HD 3870\\
    \# GPU core & 240 & 128 & 320 \\
    GPU Core Clock (MHz)&602&575&775 \\
    GPU Memory Clock (MHz)&1107&900&2250 \\
    GPU Memory Bandwidth (GB/s)& 141.7& 86.4 &72.0\\
    GPU Memory Capacity (MB)& 1024 & 768 & 512\\
    CPU&Intel Core2 Quad Q6600 & Intel Core2 Quad Q6600 & Intel Pentium 4 540 \\
    CPU Clock (MHz)&2400&2400&3200 \\
    \# CPU core& 4 & 4 & 2\\
    CPU Memory Capacity (MB)& 2048 & 2048 & 1024\\
   \hline
 \hline
\noalign{\smallskip}
  \end{tabular}
  }}
\end{table}



\section{Micro-benchmark}

We have implemented the following six real-world applications for
evaluating the MapReduce framework.

{\bf String Match (SM):} Each Map task searches a portion of the
input file to check whether the target string is in the portion.
Neither the {\em Group} nor the {\em Reduce} stage is needed.
%The three data sets include 5 million, 10 million, and 15 million text lines to match respectively.

{\bf Matrix Multiplication (MM):} Matrix multiplication is used
intensively in analyzing the relationship of two documents. Given two
matrices $M$ and $N$, each Map task computes multiplication for a
row from $M$ and a column from $N$. It outputs the pair of the row
ID and the column ID as the key and the corresponding result as the
value. Neither the {\em Group} nor the {\em Reduce} stage is needed.
%The three data sets include about 65 thousand, 262 thousand, and 1 million pairs of row and column multiplications respectively.


{\bf Black-Scholes:} Black-Scholes model~\cite{Black1973} is used for calculating the price for European options according to a partial differential equation.
For each option, a Map task computes the prices for the call and put prices of an option, and emits a structure containing the price of the option call and the price of the option put as the key, and the option id as the value. The Group stage is to rank the price of option calls. No Reduce stage is needed.


{\bf Similarity Score (SS):} It is used in web document clustering.
The characteristics of a document are represented using a feature
vector of floating point numbers. Given two document features,
$\vec{a}$ and $\vec{b}$, the similarity score between these two
documents is defined to be $\frac{\vec{a}\cdot
\vec{b}}{|\vec{a}|\cdot |\vec{b}|}$. SS computes the pair-wise
similarity score for a set of documents. Each Map task computes the
similarity score for two documents. It outputs the intermediate pair
with the score as the key and the pair of the two document IDs as the
value. The {\em Group} stage is required to rank the pair-wise similarity
scores and no {\em Reduce} stage is required. %The three data sets include
%about 130 thousand, 262 thousand, and 1 million pairs of document
%features for calculation respectively.

\red{
{\bf Principal component analysis (PCA):} This application computes the mean vector and the covariance matrix of a set of points in the first two steps in PCA.
The input data is stored in a matrix.
The whole process contains two MapReduce invocations in a chain.
The first MapReduce procedure is to find the mean for each row in the matrix, and the second is to calculate the covariance matrix.
Neither Group nor Reduce stage is needed in the first MapReduce invocation. A Map task computes the mean for a row. In the second invocation, each Map task is to calculate the covariance of two rows.
The Group stage is required to sort the row-pairs by row IDs.
No Reduce phase is needed.
}


{\bf Monte Carlo (MC):} Monte Carlo~\cite{Boyle1977} is used to compute option pricing in financial engineering.
The Monte Carlo numeric integration is to mathematically estimate the expectation of the price of option call.
Each Map task is to compute the expected value of a random sample for an option, and to emit the option ID as the key, while the expected value of the random sample as the value.
The Group stage and the Reduce stage are required to calculate the mean of all the samples for each option.
In this application, all the options have the same number of samples, and the intermediate results are ordered by option ID already. Mars does not need to perform sorting in the Group stage.

The above applications are commonly used in benchmarking MapReduce implementations in the previous studies~\cite{Chu2006, Ranger2007}. SM, MM and PCA are adopted from Phoenix suite \cite{Ranger2007},  SS is a common component in web applications,
while BS and MC are prevalent in financial engineering, and are adopted from CUDA SDK.
In particular, the workflow of these applications differ: SM and
MM only have the {\em Map} stage, BS, SS and PCA have {\em Map} and {\em
Group} stages, and MC has all the three stages. PCA has a chain of multiple MapReduce procedures, whereas other applications have
only one MapReduce invocation.

Within a single machine, we used three data sets
for each application (S, M and L) to evaluate the scalability of the
MapReduce framework.
The input for SM is textual data, and is adopted from Phoenix~\cite{Ranger2007};
The input for all the other applications contains randomly generated real numbers, ranging from zero to one.
All these input data are stored as files in the hard disk.
We summarize the size of input data for each application in Table \ref{tab:app}.

\doublerulesep 0.1pt
\begin{table}[htb]
  \centering
 \linespread{1.7}{ {\footnotesize
  \caption{The input data sizes of the micro-benchmark}\label{tab:app}
\vspace{2em}
  \begin{tabular}{cp{3.0cm}p{3.0cm}p{3.0cm}}
  \hline
\noalign{\smallskip}
  \textbf{Applications} &  \textbf{Small} & \textbf{Medium} & \textbf{Large}\\
\noalign{\smallskip}
  \hline
  String Match  & size: 55MB & size: 105MB  & size: 160MB  \\
  Matrix Multiplication  & 256x256  &  512x512  &  1024x1024  \\
  Black-Scholes  & \# option: 1,000,000  & \# option: 3,000,000  & \# option: 5,000,000  \\
  Similarity Score  & \# feature: 128, \# documents: 512  & \# feature: 128, \# documents: 1024  & \# feature: 128, \# documents: 2048  \\
  PCA  & 1000x256  & 2000x256 &  4000x256  \\
  Monte Carlo & \# option: 500, \# samples per option: 500  &  \# option: 500, \# samples per option: 2500  &  \# option: 500, \# samples per option: 5000 \\
 \hline
\noalign{\smallskip}
  \end{tabular}
  }}
\end{table}


%With the micro benchmarks, we have compared the performance and programmability of the MapReduce frameworks between the CPU and the GPU. The third party MapReduce on the CPU is the latest release of Phoenix in version 2.0.0.  As for native implementation, we have implemented the applications directly on CUDA and pthreads.  

{\bf Metrics.} The wall time is the major metric for the
performance evaluation. \red{We measure the elapsed time of each
application from reading data from the disk till generating results
in the main memory.} We ran each experiment five times and report the
average value. The variation of elapsed time between runs is negligible.
\red{The performance speedup on A over B is defined as the running time of B divided by the running time of A.
The performance slowdown on A over B is defined as the running time of A divided by the running time of B.}

We use the number of code lines written by the user as the metric on
comparing the programmability of different MapReduce implementations
as well as the native implementation with CUDA and Brook+. Note that we exclude comments and empty lines from the code size counting.

\section{Results on a Single Machine}
On a single machine, we have compared the performance and
programmability of the MapReduce frameworks between the CPU and the
GPU. \red{We have implemented the six applications on MarsCUDA, MarsCPU, and the latest release of Phoenix in version 2.0.0.}
We have also implemented the applications directly on CUDA and pthreads respectively,
including thread configuration, data distribution, task execution,  buffer management, and various memory optimizations.

We present the results on the NVIDIA GPU in detail, and briefly
present the results on the AMD GPU, mainly demonstrating the
feasibility.
\\\\
{\em 1. Results on MarsCUDA and MarsCPU}

{\bf Programmability.} Table \ref{tab:codesize} shows the comparison
of user code size, for implementing the micro-benchmark
with MarsCUDA, MarsCPU, Phoenix, and CUDA. By design, the code sizes
with MarsCUDA are the same as those with MarsCPU. In general, the
applications with MarsCPU have a smaller code size to those with
Phoenix. Phoenix needs additional code to tune the runtime performance, for example, to setup cache sizes and data chunk size, and to specify the partition and locator functions that Mars does not require. 
If the {\em Group} stage is required, applications like
SS with MarsCUDA have a much smaller code size than that
is manually written using CUDA, due to an optimized but lengthy
group function on CUDA. \red{The user code size of MarsCUDA is up to 7 times smaller than that of the native implementation with CUDA.} 
For Matrix Multiplication, CUDA have a smaller code size, because MarsCUDA requires additional code to prepare the input key/value pairs, while the native CUDA implementation does not. 

\doublerulesep 0.1pt
\begin{table}[htb]
  \centering
 \linespread{1.7}{ {\footnotesize
  \caption{Comparison of application code size on MarsCPU, MarsCUDA, Phoenix, and  CUDA.}\label{tab:codesize}
\vspace{2em}
  \begin{tabular}{cccc}
  \hline
\noalign{\smallskip}
  \textbf{Applications} & \textbf{Phoenix} & \textbf{MarsCUDA/MarsCPU} & \textbf{CUDA} \\
\noalign{\smallskip}
  \hline
    String Match & 206 &  147 & 157 \\
  Matrix Multiplication & 178 & 72 & 68\\
  Black-Scholes & 199 & 147 & 721 \\
  Similarity Score & 125 & 82 & 615 \\
  Principal component analysis & 297 &  168 & 583 \\
  Monte Carlo & 251 &  203& 359 \\
  \hline
  \end{tabular}
  }}
\end{table}

\red{
{\bf Overall performance on MapReduce.}  We conducted the performance evaluation of MarsCUDA and MarsCPU on PC A by comparing with Phoenix. Figure~\ref{fig:overall} shows the overall performance comparison. Both MarsCUDA and MarsCPU outperform Phoenix for the
six applications, due to the general lock-free design of Mars.
}

\red{
The overall
performance of MarsCPU is generally better than
that of Phoenix, achieving a speedup of up to 25.9x. Applications written using Phoenix always have a {\em Reduce} stage, whereas using ours they may not have.
Phoenix maintains a global 2D array of pointers to keys array. Each keys array is in essence a contiguous buffer as a bucket for hashing, and is sorted by insertion sort when a new key arrives.
Such design incurs two serious performance bottlenecks. First, lock-based synchronization is needed.
Second, lots of memory buffer movements (calling {\em memmove()}) are required for insertion sort in the static array.
In contrast, the design of Mars is lock-free and each Map task or Reduce task has deterministic output buffer sizes and writing positions,
so neither lock nor memory management overhead would be introduced.
In particular, BS and SS that require to rank distinct real numbers are over 10x slower on Phoenix than on MarsCPU.
That is because Phoenix has to deploy millions of identity reduce tasks for these two applications. Our profiling results obtained from Intel VTune show that over 99\% of the total execution time of BS and SS on Phoenix is contributed to the {\em memmove()} operations in the Reduce stage.
}

\begin{figure*}[ht]
\centerline{ \subfigure[Performance speedup on MarsCPU over Phoenix]{
  \includegraphics[width=0.5\textwidth]{figure/MarsCPU_Phoenix.eps}
\label{fig:marscpu_phoenix}}
\hfill
\subfigure[Performance speedup on MarsCUDA over MarsCPU (The entire MapReduce)]{
  \includegraphics[width=0.5\textwidth]{figure/MarsGPU_MarsCPU.eps}
\label{fig:marsgpu_marscpu}}
} 

\centerline{ 
\subfigure[Performance speedup on MarsCUDA over MarsCPU (On large dataset, Map \& Reduce stages only)]{
  \includegraphics[width=0.5\textwidth]{figure/kernel.eps}
\label{fig:kernel}}
}
\caption{Performance evaluation for MarsCPU and MarsCUDA on the micro-benchmark} \label{fig:overall}
\end{figure*}

\red{
As shown in \ref{fig:kernel}, MarsCUDA utilizes the GPU hardware to accelerate the Map and Reduce stages for the 6 applications, and outperforms MarsCPU in the two stages by 21x on average, and up to 40.9x.
Please note that, this speedup is obtained without specific performance tuning on the GPU code, e.g., exploiting local memory.
When it turns to the overall performance, MarsCUDA has a 10x speedup over MarsCPU for MM,  and 6x for MC, but not so impressive speedup for the other applications (Figure \ref{fig:marsgpu_marscpu}).
}

In order to figure out the source of slowdown in overall speedup, we further investigate the time breakdown of each application on the large data set for both MarsCUDA and MarsCPU. We divide the total execution time into four components,
including the time for 1) preprocessing input data (``Preprocess"), including input file I/O, generating key/value pairs, and transfering data from main memory to device memory, 2) the
{\em Map} stage (``Map"), 3) the {\em Group} stage (``Group"), and 4) the {\em
Reduce} stage (``Reduce"). 
MarsCUDA generally has a larger portion of preprocess time, involving key-val pair preparation and PCI-E I/O. In addition, the GPU-based Group stage has limited speedup over the CPU-based. We use Amdahl's law to explain this speedup involving parallel and sequential executions. Take SM for example. Although the GPU accelerates the Map phase by 20 times, the Map only takes up some 25\% in MarsCPU. According to Amdahl's law, the theoretical speedup of MarsCUDA over MarsCPU is at most 1.3. Our measurement is close to this theoretical speedup.
The preprocess is possible to be parallelized on the multi-core CPU for MarsCUDA runtime. 
However, we leave the parallelization decision to programmers, for the consideration that the runtime system can support more general purpose applications. 

\begin{figure}[h]
\centerline{ \subfigure[Time breakdown of
MarsCUDA]{\includegraphics[width=0.50\linewidth]{figure/MarsGPU_Timebreakdown.eps}
\label{fig:timebreakdowngpu}} \hfill \subfigure[Time breakdown of
MarsCPU]{\includegraphics[width=0.50\linewidth]{figure/MarsCPU_Timebreakdown.eps}
\label{fig:timebreakdowncpu}} } \caption{Time breakdown of MarsCUDA
and MarsCPU on the micro-benchmark} \label{fig:timebreakdown}
\end{figure}

{\bf Scaling.} We used the clock rate scaling tool
NVClock~\footnote{http://www.linuxhardware.org/nvclock/} to vary the
NVIDIA GPU's core clock rate and memory clock rate, in order to
evaluate the impact of hardware capability on MarsCUDA. Figures~\ref{fig:corerate} and~\ref{fig:memoryrate} show the
performance result of the six applications running on MarsCUDA with
the large data set.

In general, most applications (except for SM) on MarsCUDA are
sensitive to both core clock rate and memory clock rate.
This result indicates that MarsCUDA can scale well as the GPU evolves.
SM is not sensitive to the hardware scaling, since its GPU computation time is relatively small (as shown in Figure~\ref{fig:timebreakdowngpu}).

\begin{figure}[ht]
\centerline{ \subfigure[Baseline: Running at 100 MHz core clock rate. Memory clock rate: fixed to 1100 MHz. ]{\includegraphics[width=0.50\linewidth]{figure/corerate.eps}
\label{fig:corerate}} \hfill \subfigure[Baseline: Running at 200 MHz memory clock rate. Core clock rate: fixed to 600 MHz. ]{\includegraphics[width=0.50\linewidth]{figure/memrate.eps}
\label{fig:memoryrate}} } \caption{Varying clock rates on GTX 280.} \label{fig:freq}
\end{figure}

{\bf Comparison with native implementation.}
Figure \ref{fig:marsgpu_cuda} shows the performance slowdown of the six applications on MarsCUDA over the native implementation, with large dataset.
Overall, the implementation of applications based on MarsCUDA has roughly the same performance as on CUDA.
However, MM and MC perform much poorer on MarsCUDA, mainly due to two reasons.
One reason is rooted at the potential deficiency of MapReduce compared with a native implementation, as a previous study has already demonstrated~\cite{Ranger2007}. The other reason is that MarsCUDA does not automatically exploit the local memory to improve the temporal locality due to the lack of knowledge about specific applications.
Similarly, Figure \ref{fig:marscpu_pthread} illustrates that applications on MarsCPU has roughly the same performance as on pthreads.

{\bf Comparison with other GPU implementations.}
There are two GPU-based MapReduce implementations in parallel to our work~\cite{Catanzaro2008,Linderman2008}, while the source code is not available in public. 
Therefore, we are not able to conduct empirical performance study by comparing with these two implementations. 
The peak speedup on the NVIDIA 8800 GTX GPU over on the CPU, reported by Catanzaro~\cite{Catanzaro2008}, is better than ours (150x vs 72x). However, their MapReduce runtime implementation is highly specialized for the machine learning workloads, and they compared with the sequential CPU code, while Mars is for general purpose applications, and we compared with parallel code. 
The Merge framework~\cite{Linderman2008} reports a peak speedup of about 23x on the Intel X3000 GPU over on the CPU, while their MapReduce design is targeted on Intel's GPUs, and Mars is a general design for different many-core processors. 

\begin{figure}[ht]
\centerline{ \subfigure[MarsCUDA over CUDA.]{\includegraphics[width=0.5\linewidth]{figure/MarsGPU_CUDA.eps}
\label{fig:marsgpu_cuda}} \hfill \subfigure[MarsCPU over pthreads.]{\includegraphics[width=0.5\linewidth]{figure/MarsCPU_pthread.eps}
\label{fig:marscpu_pthread}}
} 
\centerline{ 
\subfigure[MarsBrook over Brook+.]{\includegraphics[width=0.5\linewidth]{figure/MarsBrook_brook.eps}
\label{fig:marsbrook_brook}} 
}\caption{The performance slowdown of Mars over native implementations.}

\label{fig:slowdown}
\end{figure}

{\em 2. Results on MarsBrook}

Due to the limitation of Brook+, we have developed only two
numerical applications (i.e., MM and SS) on MarsBrook. Table
\ref{tab:brookcodesize} shows the code size of applications written
in MarsBrook compared with the native implementation in Brook+. The
result is consistent with the comparison between MarsCUDA and the
native CUDA implementation. For example, the native implementation
of SS has a much larger code size than that on MarsBrook,
since SS requires a {\em Group} stage.

\doublerulesep 0.1pt
\begin{table}[htb]
  \centering
 \linespread{1.7}{ {\footnotesize
  \caption{Comparison on code sizes of MM and SS using MarsBrook and Brook+.}\label{tab:brookcodesize}
\vspace{2em}
  \begin{tabular}{ccc}
  \hline
\noalign{\smallskip}
  \textbf{Applications} & \textbf{MarsBrook} & \textbf{Brook+} \\
\noalign{\smallskip}
  \hline
  MM & 66 & 93 \\
  SS & 66 & 611  \\
  \hline
\noalign{\smallskip}
  \end{tabular}
  }}
\end{table}

Figure \ref{fig:marsbrook_brook} shows the performance slowdown of
two applications by using MarsBrook over the native implementation.
The implementation on top of MarsBrook is up to twice slower than
the native implementation, which is the price to pay for the user code size reduction.
\\\\
{\em 3. Results on GPU/CPU co-processing of Mars}

We used MarsCUDA and MarsCPU as two components in the co-processing.
Figure \ref{fig:coprocess} shows the performance speedup of the GPU/CPU co-processing module over MarsCUDA, MarsCPU, and Phoenix, on the large dataset.
Overall, co-processing utilizes the computation power of both the CPU and the GPU, and yields a considerable performance improvement over using MarsCPU or Phoenix on a CPU.
However, the speedup of using co-processing over using standalone MarsCUDA is limited.

The workload dispatching between MarsCUDA and MarsCPU in co-processing mainly depends on the performance comparison between the CPU processing and the GPU processing.
The theoretical speedup of co-processing over MarsCUDA would be $(S + 1) / S$, where $S$ is the speedup of using MarsCUDA over using MarsCPU.
For example, if the speedup $S$ is 10, then using co-processing would only outperform using standalone MarsCUDA by a factor of $\frac{10+1}{10} = 1.1$.
Therefore, for compute-intensive applications MM, BS, SS, MC, and PCA, using co-processing cannot boost the performance considerably over using the standalone MarsCUDA.
For SM that spends most time in preprocessing, using co-processing can hardly achieve the theoretical speedup $\frac{1+1}{1} = 2$.
Nevertheless, applications using co-processing of MarsCUDA and MarsCPU still outperforms Phoenix with a speedup of 24 times on average, and 72 times at maximum.

\begin{figure}[h]
 \centering
 \includegraphics[width=0.65\textwidth]{figure/coprocess.eps}
 \caption{Performance speedup of GPU/CPU co-processing module over MarsCUDA, MarsCPU, and Phoenix.}\label{fig:coprocess}
\end{figure}


\section{Results on MarsHadoop}

We experimented MM on MarsHadoop. We configured Hadoop on
PC A and PC B: PC A as the master node, while PC A itself and PC B
as slave nodes.


Figure \ref{fig:hadoopspeedup} shows the performance speedup of
MarsHadoop over the native Hadoop implementation on MM. As the
matrix size varied, MarsHadoop is up to 2.8 times faster than the
native Hadoop implementation. We further examine the time breakdown
in the slave node, and the results are shown in Figure
\ref{fig:hadoopmmbreakdown}. As the matrix size increases, the ratio
for the computation time grows, indicating that Mars starts to help.
The disk I/O is mainly due to the extra I/O caused by Hadoop
streaming.

\begin{figure}[h]
\centerline{
\subfigure[Performance speedup on MarsHadoop over native Hadoop.]{\includegraphics[width=0.50\linewidth]{figure/speeduphadoop.eps}
\label{fig:hadoopspeedup}}
\hfill
\subfigure[Time breakdown on MarsHadoop.]{\includegraphics[width=0.5\linewidth]{figure/breakdownhadoop.eps}
\label{fig:hadoopmmbreakdown}} } \caption{Matrix Multiplication on MarsHadoop}
\end{figure}


%\chapter{Conclusion}\label{sec-conclusion}
Graphics processors have become an efficient accelerator for
high-performance computing. This thesis proposes Mars, which
harnesses the GPU computation power and high memory bandwidth to
accelerate MapReduce frameworks. Mars is applicable to run on
NVIDIA GPUs, AMD GPUs, multi-core CPUs, and Hadoop-based distributed systems. 
%and can be easily ported to other parallel systems, due to its lock-free design and simple array data structure. 
%In addition, Mars is flexible enough to support a variety of applications efficiently, because of the support of customized workflow. 
Our empirical studies show that Mars
improves the programmability of both the NVIDIA and the AMD GPUs,
and \red{the GPU-CPU co-processing of Mars on an NVIDIA GTX280 GPU and an Intel quad-core CPU outperformed Phoenix, the state-of-the-art MapReduce on the multi-core CPU with a speedup of up to 72 times and 24 times on average.} Additionally, integrating Mars into Hadoop enabled GPU acceleration for a network of PCs.

The code and documentation of Mars can be found at
http://www.cse.ust.hk/gpuqp/.



%%%%%%%%%%%%%%%%%%%%%%%%%%%%%%%%%%%%%%%%%%%%%%%%%%%%%%%%%%%%%%%%%%%%%%%%%
%                                                                       %
%      9) BIBLIOGRAPHY                                                  %
%                                                                       %
% This example uses bibtex to generate the required Bibliography. Refer %
% to the % the file ustthesis_test.bib for the entries of the           %
% Bibliography. Note that only the cited entries are printed.           %
%                                                                       %
% If BibTeX is not used to typeset the bibliography, replace the        %
% following line with the \begin{thebibliography} and \end{bibliography}%
% commands (the "thebibliography" environment) to process the           %
% Bibliography.                                                         %
%                                                                       %
%%%%%%%%%%%%%%%%%%%%%%%%%%%%%%%%%%%%%%%%%%%%%%%%%%%%%%%%%%%%%%%%%%%%%%%%%

%%%%%%%%%%%%%%%%%%%%%%%%%%%%%%%%%%%%%%%%%%%%%%%%%%%%%%%%%%%%%%%%%%%%%%%%%
%                                                                       %
% The recommended bibliography style is the IEEE bibliography style.    %
% "ustbib" defines the IEEE bibliography standard with the added        %
% ability of sorting the items by name of author.                       %
%                                                                       %
% If you are not using BibTeX to process your Bibliography, comment out %
% the following line.                                                   %
%                                                                       %
%%%%%%%%%%%%%%%%%%%%%%%%%%%%%%%%%%%%%%%%%%%%%%%%%%%%%%%%%%%%%%%%%%%%%%%%%

\bibliographystyle{plain}

\bibliography{ref}
% Please run "bibtex ustthesis_test" before the bibliography can be
% included.

%%%%%%%%%%%%%%%%%%%%%%%%%%%%%%%%%%%%%%%%%%%%%%%%%%%%%%%%%%%%%%%%%%%%%%%%%
%                                                                       %
%     10) APPENDIX (If Any)                                              %
%                                                                       %
% \appendix command marks the beginning of the APPENDIX part of the     %
% Thesis. The usual \chapter command is used for the different chapters %
% of the Appendix.                                                      %
%                                                                       %
%%%%%%%%%%%%%%%%%%%%%%%%%%%%%%%%%%%%%%%%%%%%%%%%%%%%%%%%%%%%%%%%%%%%%%%%%


%%%%%%%%%%%%%%%%%%%%%%%%%%%%%%%%%%%%%%%%%%%%%%%%%%%%%%%%%%%%%%%%%%%%%%%%%
%                                                                       %
%     11) BIOGRAPHY (Optional)                                          %
%                                                                       %
% \biography and \endbiography are used to define the optional          %
% Biography of the author of the Thesis.                                %
%                                                                       %
%%%%%%%%%%%%%%%%%%%%%%%%%%%%%%%%%%%%%%%%%%%%%%%%%%%%%%%%%%%%%%%%%%%%%%%%%

% \biography
% The biography of the student is ALSO optional.
% \endbiography

\end{document}
